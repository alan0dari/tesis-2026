\documentclass[12pt,letterpaper,twoside]{book}
\usepackage[utf8]{inputenc}
\usepackage[spanish]{babel}
\usepackage{amsmath}
\usepackage{amsfonts}
\usepackage{amssymb}
\usepackage{graphicx}
\usepackage[left=3cm,right=2.5cm,top=2.5cm,bottom=2.5cm]{geometry}
\usepackage{setspace}
\usepackage{fancyhdr}
\usepackage{cite}
\usepackage{url}
\usepackage{hyperref}
\usepackage{algorithm}
\usepackage{algorithmic}
\usepackage{listings}
\usepackage{xcolor}

% Configuración de hiperenlaces
\hypersetup{
    colorlinks=true,
    linkcolor=blue,
    filecolor=magenta,      
    urlcolor=cyan,
    citecolor=blue,
    pdftitle={Framework CLAHE-MCDM},
    pdfauthor={Autor},
}

% Configuración de código
\lstset{
    basicstyle=\ttfamily\small,
    breaklines=true,
    frame=single,
    language=Python,
    showstringspaces=false,
    commentstyle=\color{green!50!black},
    keywordstyle=\color{blue},
    stringstyle=\color{red}
}

% Configuración de encabezados
\pagestyle{fancy}
\fancyhf{}
\fancyhead[LE,RO]{\thepage}
\fancyhead[RE]{\leftmark}
\fancyhead[LO]{\rightmark}

% Espaciado
\onehalfspacing

% Información del documento
\title{Framework basado en aplicación de métodos de decisión multicriterio para selección de imagen radiográfica mejorada con optimización multiobjetivo}
\author{Nombre del Autor}
\date{2026}

\begin{document}

% Portada
\begin{titlepage}
    \centering
    \vspace*{1cm}
    
    {\LARGE\bfseries UNIVERSIDAD [NOMBRE]\par}
    \vspace{0.5cm}
    {\Large FACULTAD DE [NOMBRE]\par}
    \vspace{0.5cm}
    {\Large PROGRAMA DE [NOMBRE]\par}
    
    \vspace{3cm}
    
    {\Huge\bfseries Framework basado en aplicación de métodos de decisión multicriterio para selección de imagen radiográfica mejorada con optimización multiobjetivo\par}
    
    \vspace{3cm}
    
    {\Large\itshape Nombre del Autor\par}
    
    \vfill
    
    {\large Tesis presentada como requisito para optar al título de\\
    [Maestría en ...]\par}
    
    \vspace{1cm}
    
    {\large Director: Dr./Mg. [Nombre del Director]\par}
    
    \vfill
    
    {\large Ciudad, País\par}
    {\large 2026\par}
\end{titlepage}

% Página de aprobación
\chapter*{Nota de Aceptación}
\addcontentsline{toc}{chapter}{Nota de Aceptación}

\vspace{3cm}

\noindent
\rule{10cm}{0.5pt}\\
Firma del Director de Tesis\\
Dr./Mg. [Nombre]

\vspace{1.5cm}

\noindent
\rule{10cm}{0.5pt}\\
Firma del Jurado\\
Dr./Mg. [Nombre]

\vspace{1.5cm}

\noindent
\rule{10cm}{0.5pt}\\
Firma del Jurado\\
Dr./Mg. [Nombre]

\vspace{2cm}

\noindent
Ciudad, [Fecha]

\newpage

% Dedicatoria
\chapter*{Dedicatoria}
\addcontentsline{toc}{chapter}{Dedicatoria}

\vspace{3cm}

\begin{flushright}
\textit{[Texto de dedicatoria]}
\end{flushright}

\newpage

% Agradecimientos
\chapter*{Agradecimientos}
\addcontentsline{toc}{chapter}{Agradecimientos}

[Texto de agradecimientos]

\newpage

% Resumen
\chapter*{Resumen}
\addcontentsline{toc}{chapter}{Resumen}

Este trabajo presenta un framework innovador para la mejora de ortopantomografías mediante la integración de técnicas de procesamiento de imágenes, optimización multiobjetivo y métodos de decisión multicriterio. El framework utiliza CLAHE (Contrast Limited Adaptive Histogram Equalization) para el realce de contraste, SMPSO (Speed-constrained Multi-objective Particle Swarm Optimization) para la optimización de parámetros, y ocho métodos MCDM (SMARTER, TOPSIS, Bellman-Zadeh, PROMETHEE II, GRA, VIKOR, CODAS, MABAC) para la selección de la mejor imagen mejorada.

Los resultados experimentales demuestran que el framework propuesto mejora significativamente la calidad visual de las ortopantomografías, optimizando simultáneamente múltiples métricas objetivas (entropía, SSIM y VQI). Los métodos MCDM permiten seleccionar la solución más apropiada del Frente de Pareto según las preferencias del usuario, facilitando la toma de decisiones en aplicaciones clínicas.

\textbf{Palabras clave:} CLAHE, Optimización Multiobjetivo, SMPSO, MCDM, Ortopantomografía, Procesamiento de Imágenes Médicas, Frente de Pareto.

\newpage

% Abstract
\chapter*{Abstract}
\addcontentsline{toc}{chapter}{Abstract}

This work presents an innovative framework for enhancing orthopantomograms through the integration of image processing techniques, multi-objective optimization, and multi-criteria decision-making methods. The framework uses CLAHE (Contrast Limited Adaptive Histogram Equalization) for contrast enhancement, SMPSO (Speed-constrained Multi-objective Particle Swarm Optimization) for parameter optimization, and eight MCDM methods (SMARTER, TOPSIS, Bellman-Zadeh, PROMETHEE II, GRA, VIKOR, CODAS, MABAC) for selecting the best enhanced image.

Experimental results demonstrate that the proposed framework significantly improves the visual quality of orthopantomograms by simultaneously optimizing multiple objective metrics (entropy, SSIM, and VQI). MCDM methods enable selection of the most appropriate solution from the Pareto Front according to user preferences, facilitating decision-making in clinical applications.

\textbf{Keywords:} CLAHE, Multi-objective Optimization, SMPSO, MCDM, Orthopantomogram, Medical Image Processing, Pareto Front.

\newpage

% Tabla de contenidos
\tableofcontents

% Lista de figuras
\listoffigures

% Lista de tablas
\listoftables

\newpage

% Capítulos
\chapter{Introducción}

\section{Contexto y Motivación}

Las ortopantomografías, también conocidas como radiografías panorámicas dentales, son herramientas diagnósticas esenciales en odontología que permiten visualizar la estructura completa de la dentadura, maxilares y estructuras adyacentes en una sola imagen. Sin embargo, estas imágenes a menudo presentan problemas de contraste debido a las limitaciones inherentes del proceso de adquisición, lo que puede dificultar la interpretación diagnóstica precisa.

La mejora del contraste en imágenes médicas es un problema fundamental en el procesamiento de imágenes, especialmente en radiología dental donde la calidad de la imagen impacta directamente en la capacidad del profesional para detectar anomalías, planificar tratamientos y realizar diagnósticos precisos.

\section{Planteamiento del Problema}

Los principales desafíos en la mejora de ortopantomografías incluyen:

\begin{itemize}
    \item \textbf{Variabilidad en la calidad de imagen:} Las condiciones de adquisición varían entre pacientes y equipos, resultando en imágenes con diferentes niveles de contraste y exposición.
    
    \item \textbf{Selección de parámetros óptimos:} Los métodos de mejora como CLAHE requieren la configuración de múltiples parámetros (región contextual, límite de contraste) cuya selección óptima es compleja.
    
    \item \textbf{Múltiples objetivos en conflicto:} La mejora debe optimizar simultáneamente varios criterios (entropía, similitud estructural, calidad visual) que pueden estar en conflicto.
    
    \item \textbf{Subjetividad en la evaluación:} La percepción de calidad visual varía entre observadores, haciendo necesario un enfoque sistemático para la selección de la mejor imagen mejorada.
\end{itemize}

\section{Objetivos}

\subsection{Objetivo General}

Desarrollar un framework integral que combine técnicas de procesamiento de imágenes, optimización multiobjetivo y métodos de decisión multicriterio para la mejora y selección automática de ortopantomografías de alta calidad.

\subsection{Objetivos Específicos}

\begin{enumerate}
    \item Implementar el algoritmo CLAHE con parámetros ajustables para la mejora de contraste en ortopantomografías.
    
    \item Desarrollar un sistema de optimización multiobjetivo basado en SMPSO para encontrar configuraciones óptimas de parámetros CLAHE.
    
    \item Integrar múltiples métricas de evaluación (entropía, SSIM, VQI) para caracterizar la calidad de las imágenes mejoradas.
    
    \item Implementar ocho métodos de decisión multicriterio (SMARTER, TOPSIS, Bellman-Zadeh, PROMETHEE II, GRA, VIKOR, CODAS, MABAC) para la selección de la mejor solución del Frente de Pareto.
    
    \item Validar el framework mediante experimentación con imágenes reales y comparación con métodos tradicionales.
    
    \item Evaluar la concordancia entre las selecciones de diferentes métodos MCDM y la percepción de profesionales médicos.
\end{enumerate}

\section{Justificación}

Este trabajo se justifica por las siguientes razones:

\subsection{Relevancia Clínica}

Las ortopantomografías son fundamentales para el diagnóstico y planificación de tratamientos odontológicos. Una mejor calidad de imagen permite:
\begin{itemize}
    \item Detección temprana de patologías
    \item Planificación precisa de implantes dentales
    \item Evaluación de estructuras maxilofaciales
    \item Reducción de errores diagnósticos
\end{itemize}

\subsection{Innovación Metodológica}

La integración de optimización multiobjetivo con múltiples métodos MCDM representa un enfoque novedoso que:
\begin{itemize}
    \item Automatiza la selección de parámetros óptimos
    \item Proporciona múltiples soluciones de compromiso
    \item Permite incorporar preferencias del usuario
    \item Ofrece robustez mediante consenso entre métodos
\end{itemize}

\subsection{Aplicabilidad Práctica}

El framework propuesto es:
\begin{itemize}
    \item Independiente del equipo de adquisición
    \item Escalable a diferentes tipos de imágenes médicas
    \item Implementable en flujos de trabajo clínicos existentes
    \item Extensible a otros problemas de optimización multiobjetivo
\end{itemize}

\section{Alcance y Limitaciones}

\subsection{Alcance}

El presente trabajo abarca:
\begin{itemize}
    \item Implementación completa del framework propuesto
    \item Experimentación con ortopantomografías
    \item Análisis comparativo de métodos MCDM
    \item Validación con profesionales médicos
    \item Documentación y código abierto
\end{itemize}

\subsection{Limitaciones}

Las limitaciones del estudio incluyen:
\begin{itemize}
    \item Enfoque específico en ortopantomografías (aunque extensible)
    \item Uso de métricas objetivas que pueden no capturar completamente la percepción humana
    \item Tiempo computacional requerido para la optimización
    \item Necesidad de validación con conjuntos de datos más amplios
\end{itemize}

\section{Estructura del Documento}

El resto de este documento está organizado de la siguiente manera:

\begin{description}
    \item[Capítulo 2:] Marco teórico sobre imágenes médicas, ortopantomografías y técnicas de mejora de contraste.
    
    \item[Capítulo 3:] Fundamentos de optimización multiobjetivo, SMPSO y métodos de decisión multicriterio.
    
    \item[Capítulo 4:] Metodología propuesta, describiendo la arquitectura del framework y sus componentes.
    
    \item[Capítulo 5:] Experimentación y resultados, incluyendo configuración experimental, análisis de resultados y validación.
    
    \item[Capítulo 6:] Conclusiones, contribuciones del trabajo y direcciones futuras de investigación.
\end{description}

\chapter{Marco Teórico: Imágenes Médicas y Ortopantomografías}

\section{Imágenes Médicas}

\subsection{Características de las Imágenes Médicas}

Las imágenes médicas son representaciones visuales de la anatomía interna del cuerpo humano obtenidas mediante diversas modalidades de adquisición. Se caracterizan por:

\begin{itemize}
    \item \textbf{Alta dimensionalidad:} Resoluciones de varios megapíxeles
    \item \textbf{Rango dinámico variable:} Dependiendo de la modalidad
    \item \textbf{Ruido inherente:} Debido al proceso de adquisición
    \item \textbf{Artefactos:} Distorsiones y anomalías del proceso de captura
\end{itemize}

\subsection{Modalidades de Imágenes Médicas}

\subsubsection{Rayos X}
Utilizan radiación ionizante para generar imágenes de estructuras óseas y tejidos densos.

\subsubsection{Tomografía Computarizada (CT)}
Genera imágenes tridimensionales mediante múltiples proyecciones de rayos X.

\subsubsection{Resonancia Magnética (MRI)}
Utiliza campos magnéticos y ondas de radio para visualizar tejidos blandos.

\subsubsection{Ultrasonido}
Emplea ondas sonoras de alta frecuencia para generar imágenes en tiempo real.

\section{Ortopantomografías}

\subsection{Definición y Principio de Funcionamiento}

La ortopantomografía, también conocida como radiografía panorámica dental, es una técnica radiológica que produce una imagen bidimensional de la dentadura completa, maxilares, articulaciones temporomandibulares y estructuras circundantes en una sola toma.

\subsubsection{Principio Físico}
\begin{itemize}
    \item Fuente de rayos X y detector se mueven sincronizadamente
    \item Captura estructuras en un arco curvo
    \item Proyección ortogonal de estructuras tridimensionales en plano 2D
\end{itemize}

\subsection{Aplicaciones Clínicas}

Las ortopantomografías se utilizan para:

\begin{enumerate}
    \item \textbf{Diagnóstico general:}
    \begin{itemize}
        \item Detección de caries
        \item Evaluación de enfermedad periodontal
        \item Identificación de quistes y tumores
    \end{itemize}
    
    \item \textbf{Planificación de tratamientos:}
    \begin{itemize}
        \item Implantes dentales
        \item Cirugía ortognática
        \item Extracción de muelas del juicio
    \end{itemize}
    
    \item \textbf{Evaluación de desarrollo:}
    \begin{itemize}
        \item Erupción dental en niños
        \item Anomalías de desarrollo
    \end{itemize}
\end{enumerate}

\subsection{Limitaciones y Desafíos}

\subsubsection{Limitaciones Técnicas}
\begin{itemize}
    \item Distorsión geométrica inherente
    \item Superposición de estructuras
    \item Variabilidad en exposición
    \item Artefactos de movimiento
\end{itemize}

\subsubsection{Desafíos de Contraste}
\begin{itemize}
    \item Bajo contraste en tejidos blandos
    \item Variabilidad en densidad ósea
    \item Efectos de endurecimiento del haz
    \item Ruido cuántico
\end{itemize}

\section{Procesamiento de Imágenes Médicas}

\subsection{Preprocesamiento}

\subsubsection{Reducción de Ruido}
Técnicas para eliminar ruido mientras se preservan bordes y detalles importantes.

\subsubsection{Normalización}
Ajuste de intensidades para estandarizar imágenes de diferentes fuentes.

\subsection{Mejora de Contraste}

\subsubsection{Ecualización de Histograma Global}
Redistribuye los niveles de intensidad para utilizar todo el rango dinámico.

\[
h(i) = \frac{n_i}{N}
\]

donde $n_i$ es el número de píxeles con intensidad $i$ y $N$ es el número total de píxeles.

\subsubsection{Ecualización de Histograma Adaptativa}
Aplica ecualización en regiones locales de la imagen.

\subsubsection{CLAHE (Contrast Limited Adaptive Histogram Equalization)}
Versión mejorada de AHE que limita la amplificación de contraste para evitar sobre-realce de ruido.

\textbf{Parámetros principales:}
\begin{itemize}
    \item $R_x, R_y$: Tamaño de la región contextual (tile size)
    \item $C$: Límite de contraste (clip limit)
\end{itemize}

\textbf{Algoritmo:}
\begin{algorithm}
\caption{CLAHE}
\begin{algorithmic}
\STATE Dividir imagen en regiones $(R_x \times R_y)$
\FOR{cada región}
    \STATE Calcular histograma local
    \STATE Aplicar límite de contraste $C$
    \STATE Redistribuir píxeles recortados uniformemente
    \STATE Ecualizar histograma
\ENDFOR
\STATE Interpolar entre regiones para eliminar bordes
\end{algorithmic}
\end{algorithm}

\section{Métricas de Evaluación de Calidad}

\subsection{Entropía de Shannon}

Mide la cantidad de información contenida en una imagen:

\[
H = -\sum_{i=0}^{L-1} p_i \log_2(p_i)
\]

donde $p_i$ es la probabilidad del nivel de gris $i$.

\textbf{Interpretación:}
\begin{itemize}
    \item Mayor entropía $\rightarrow$ Mayor información
    \item Rango: $[0, \log_2(L)]$ donde $L$ es el número de niveles de gris
\end{itemize}

\subsection{SSIM (Structural Similarity Index)}

Evalúa la similitud estructural entre dos imágenes:

\[
\text{SSIM}(x,y) = \frac{(2\mu_x\mu_y + C_1)(2\sigma_{xy} + C_2)}{(\mu_x^2 + \mu_y^2 + C_1)(\sigma_x^2 + \sigma_y^2 + C_2)}
\]

donde:
\begin{itemize}
    \item $\mu_x, \mu_y$: medias
    \item $\sigma_x, \sigma_y$: desviaciones estándar
    \item $\sigma_{xy}$: covarianza
    \item $C_1, C_2$: constantes de estabilización
\end{itemize}

\textbf{Componentes:}
\begin{itemize}
    \item Luminancia: $l(x,y)$
    \item Contraste: $c(x,y)$
    \item Estructura: $s(x,y)$
\end{itemize}

\subsection{VQI (Visual Quality Index)}

Métrica perceptual que considera:
\begin{itemize}
    \item Contraste local
    \item Nitidez (sharpness)
    \item Distribución de intensidades
    \item Ausencia de artefactos
\end{itemize}

\section{Estado del Arte}

\subsection{Métodos Tradicionales}

\subsubsection{Mejora Basada en Histograma}
\begin{itemize}
    \item Ecualización global
    \item Especificación de histograma
    \item Ecualización local
\end{itemize}

\subsubsection{Mejora Basada en Dominio de Frecuencia}
\begin{itemize}
    \item Filtrado homómrfico
    \item Filtros de realce de alta frecuencia
\end{itemize}

\subsection{Métodos Avanzados}

\subsubsection{Basados en Optimización}
\begin{itemize}
    \item Algoritmos genéticos para ajuste de parámetros
    \item Optimización por enjambre de partículas
    \item Recocido simulado
\end{itemize}

\subsubsection{Basados en Aprendizaje Profundo}
\begin{itemize}
    \item Redes neuronales convolucionales
    \item Redes adversarias generativas (GANs)
    \item Autoencoders
\end{itemize}

\subsection{Gaps en la Literatura}

A pesar de los avances, existen áreas poco exploradas:
\begin{itemize}
    \item Integración de optimización multiobjetivo con MCDM
    \item Validación sistemática con múltiples métricas
    \item Frameworks flexibles y extensibles
    \item Consideración de preferencias de usuarios expertos
\end{itemize}

\chapter{Marco Teórico: Optimización Multiobjetivo y MCDM}

\section{Optimización Multiobjetivo}

\subsection{Definición del Problema}

Un problema de optimización multiobjetivo (MOP) se define como:

\begin{align*}
\text{minimizar/maximizar} \quad & \mathbf{F}(\mathbf{x}) = [f_1(\mathbf{x}), f_2(\mathbf{x}), \ldots, f_m(\mathbf{x})]^T \\
\text{sujeto a} \quad & \mathbf{x} \in \Omega
\end{align*}

donde $\mathbf{x} \in \mathbb{R}^n$ es el vector de variables de decisión y $\Omega$ es el espacio de búsqueda factible.

\subsection{Dominancia de Pareto}

Una solución $\mathbf{x}_1$ domina a $\mathbf{x}_2$ ($\mathbf{x}_1 \prec \mathbf{x}_2$) si:
\begin{itemize}
    \item $f_i(\mathbf{x}_1) \leq f_i(\mathbf{x}_2)$ para todo $i \in \{1,\ldots,m\}$
    \item $f_j(\mathbf{x}_1) < f_j(\mathbf{x}_2)$ para al menos un $j \in \{1,\ldots,m\}$
\end{itemize}

\subsection{Frente de Pareto}

El conjunto de soluciones no dominadas forma el \textit{Frente de Pareto}:
\[
\mathcal{P} = \{\mathbf{x} \in \Omega \mid \nexists \mathbf{x}' \in \Omega : \mathbf{x}' \prec \mathbf{x}\}
\]

\section{SMPSO (Speed-constrained Multi-objective PSO)}

\subsection{Particle Swarm Optimization (PSO)}

PSO es un algoritmo de optimización inspirado en el comportamiento social de aves y peces.

\textbf{Ecuaciones de actualización:}
\begin{align}
v_i^{t+1} &= w \cdot v_i^t + c_1 r_1 (p_i - x_i^t) + c_2 r_2 (p_g - x_i^t) \\
x_i^{t+1} &= x_i^t + v_i^{t+1}
\end{align}

\subsection{Extensión Multiobjetivo: SMPSO}

SMPSO introduce:
\begin{enumerate}
    \item \textbf{Restricción de velocidad:} 
    \[
    v_{\max} = \frac{x_{\max} - x_{\min}}{2}
    \]
    
    \item \textbf{Archivo externo:} Mantiene soluciones no dominadas
    
    \item \textbf{Mutación polinomial:} Mantiene diversidad
    
    \item \textbf{Selección de líderes:} Basada en crowding distance
\end{enumerate}

\section{Métodos de Decisión Multicriterio (MCDM)}

\subsection{Problema de Decisión}

Dado un Frente de Pareto con $n$ alternativas y $m$ criterios, seleccionar la mejor alternativa según preferencias del decisor.

\subsection{SMARTER}

\textbf{Idea:} Función de utilidad aditiva con pesos por ranking.

\[
U(A_i) = \sum_{j=1}^m w_j \cdot v_{ij}
\]

Pesos ROC (Rank Order Centroid):
\[
w_j = \frac{1}{m} \sum_{k=j}^m \frac{1}{k}
\]

\subsection{TOPSIS}

\textbf{Idea:} Seleccionar alternativa más cercana a ideal positivo y más lejana de ideal negativo.

\[
C_i = \frac{D_i^-}{D_i^+ + D_i^-}
\]

\subsection{Bellman-Zadeh}

\textbf{Idea:} Decisión difusa como intersección de objetivos.

\[
\mu_D(A_i) = \min_{j} \mu_j(A_i)
\]

\subsection{PROMETHEE II}

\textbf{Idea:} Flujos de preferencia basados en comparaciones pareadas.

\[
\phi(a) = \phi^+(a) - \phi^-(a)
\]

\subsection{GRA (Grey Relational Analysis)}

\textbf{Idea:} Coeficientes de relación gris con respecto a secuencia ideal.

\[
\xi_{ij} = \frac{\Delta_{\min} + \zeta \Delta_{\max}}{\Delta_{ij} + \zeta \Delta_{\max}}
\]

\subsection{VIKOR}

\textbf{Idea:} Índice de compromiso considerando utilidad grupal y arrepentimiento.

\[
Q_i = v \frac{S_i - S^*}{S^- - S^*} + (1-v) \frac{R_i - R^*}{R^- - R^*}
\]

\subsection{CODAS}

\textbf{Idea:} Distancias Euclidiana y Taxicab desde ideal negativo.

\[
H_i = E_i + \sum_k \Psi_{ik} T_i
\]

\subsection{MABAC}

\textbf{Idea:} Distancia al área de aproximación de borde.

\[
S_i = \sum_j (Q_{ij} - G_j)
\]

donde $G_j = (\\prod_i Q_{ij})^{1/n}$ es el BAA.

\section{Métricas de Calidad del Frente de Pareto}

\subsection{Hipervolumen}

Volumen del espacio objetivo dominado por el Frente:
\[
HV = \text{volumen}(\bigcup_{i=1}^{|\mathcal{P}|} \{y \mid y \preceq \mathbf{F}(x_i)\})
\]

\subsection{Spacing}

Uniformidad de distribución:
\[
S = \sqrt{\frac{1}{n}\sum_{i=1}^n (d_i - \bar{d})^2}
\]

\subsection{Spread}

Extensión del Frente:
\[
\Delta = \frac{d_f + d_l + \sum_{i=1}^{n-1}|d_i - \bar{d}|}{d_f + d_l + (n-1)\bar{d}}
\]

\section{Integración de Optimización y MCDM}

\subsection{Flujo General}

\begin{enumerate}
    \item Definir función multiobjetivo
    \item Ejecutar SMPSO para obtener Frente de Pareto
    \item Aplicar múltiples métodos MCDM
    \item Analizar consenso entre métodos
    \item Seleccionar solución final
\end{enumerate}

\subsection{Ventajas de la Integración}

\begin{itemize}
    \item Exploración completa del espacio de soluciones
    \item Múltiples perspectivas de selección
    \item Robustez mediante consenso
    \item Flexibilidad en preferencias
\end{itemize}

\chapter{Metodología Propuesta}

\section{Arquitectura del Framework}

El framework propuesto consta de cinco módulos principales:

\begin{enumerate}
    \item \textbf{Módulo de Procesamiento CLAHE}
    \item \textbf{Módulo de Métricas de Evaluación}
    \item \textbf{Módulo de Optimización Multiobjetivo}
    \item \textbf{Módulo de Métodos MCDM}
    \item \textbf{Módulo de Visualización}
\end{enumerate}

\section{Flujo de Trabajo}

\begin{algorithm}
\caption{Framework CLAHE-MCDM}
\begin{algorithmic}
\STATE \textbf{Entrada:} Ortopantomografía $I$
\STATE \textbf{Salida:} Imagen mejorada óptima $I^*$
\STATE
\STATE // Fase 1: Optimización
\STATE Definir función objetivo $\mathbf{F}(R_x, R_y, C) = [Entropía, SSIM, VQI]$
\STATE Ejecutar SMPSO con $n$ partículas, $t$ iteraciones
\STATE Obtener Frente de Pareto $\mathcal{P}$
\STATE
\STATE // Fase 2: Selección MCDM
\STATE Construir matriz de decisión desde $\mathcal{P}$
\FOR{cada método MCDM $m \in \{$SMARTER, TOPSIS, ..., MABAC$\}$}
    \STATE Aplicar método $m$
    \STATE Obtener ranking y mejor solución $s_m$
\ENDFOR
\STATE
\STATE // Fase 3: Consenso
\STATE Analizar concordancia entre métodos
\STATE Seleccionar solución final $I^*$
\STATE \textbf{Retornar} $I^*$
\end{algorithmic}
\end{algorithm}

\section{Implementación}

\subsection{Tecnologías Utilizadas}

\begin{itemize}
    \item \textbf{Lenguaje:} Python 3.8+
    \item \textbf{Librerías principales:}
    \begin{itemize}
        \item NumPy, SciPy: Computación científica
        \item OpenCV: Procesamiento de imágenes
        \item scikit-image: Métricas de calidad
        \item Matplotlib: Visualización
        \item pandas: Manejo de datos
    \end{itemize}
\end{itemize}

\subsection{Estructura del Código}

\begin{lstlisting}[language=bash]
src/
├── clahe/          # Procesamiento CLAHE
├── metrics/        # Métricas (entropía, SSIM, VQI)
├── optimization/   # SMPSO y Pareto
├── mcdm/          # 8 métodos MCDM
└── utils/         # Utilidades
\end{lstlisting}

\section{Configuración Experimental}

\subsection{Parámetros de CLAHE}

El espacio de búsqueda de CLAHE está definido por tres parámetros que determinan 
el comportamiento del algoritmo de mejora de contraste:

\begin{table}[h]
\centering
\begin{tabular}{|l|c|l|l|}
\hline
\textbf{Parámetro} & \textbf{Rango} & \textbf{Valores} & \textbf{Descripción} \\
\hline
$R_x$ & [2, 64] & 63 enteros & Regiones contextuales (filas) \\
$R_y$ & [2, 64] & 63 enteros & Regiones contextuales (columnas) \\
$C$ (clip limit) & [1.0, 4.0] & Continuo & Límite de contraste local \\
\hline
\end{tabular}
\caption{Rangos de parámetros CLAHE para optimización}
\label{tab:clahe_params}
\end{table}

El espacio de búsqueda total comprende aproximadamente:
\begin{equation}
|\Omega| = 63 \times 63 \times 301 \approx 1,195,000 \text{ combinaciones}
\end{equation}

donde 301 representa la discretización del clip limit con resolución de 0.01.

\subsection{Parámetros de SMPSO}

La configuración del algoritmo SMPSO se seleccionó para garantizar una exploración
exhaustiva del espacio de búsqueda, siguiendo las recomendaciones de 
\citet{coello2007evolutionary} para optimización multiobjetivo:

\begin{table}[h]
\centering
\begin{tabular}{|l|c|l|}
\hline
\textbf{Parámetro} & \textbf{Valor} & \textbf{Justificación} \\
\hline
Partículas ($n$) & 500 & $\approx 10\text{-}20 \times$ dimensiones \\
Iteraciones ($t$) & 250 & Convergencia completa \\
$c_1$ (cognitivo) & 1.5 & Balance exploración/explotación \\
$c_2$ (social) & 1.5 & Valor estándar \\
$w_{max}$ (inercia inicial) & 0.9 & Mayor exploración inicial \\
$w_{min}$ (inercia final) & 0.4 & Mayor explotación final \\
Tamaño archivo & 500 & $\geq n$ para preservar diversidad \\
Prob. mutación & 0.1 & Mantener diversidad \\
\hline
\end{tabular}
\caption{Configuración SMPSO para cobertura exhaustiva}
\label{tab:smpso_config}
\end{table}

Con esta configuración, el número total de evaluaciones de la función objetivo es:
\begin{equation}
\text{Evaluaciones} = n \times t = 500 \times 250 = 125,000
\end{equation}

Esto representa aproximadamente el 10.5\% del espacio de búsqueda total,
lo cual es suficiente para que SMPSO converja al verdadero Frente de Pareto 
\citep{nebro2009smpso}.

\subsection{Pesos de Criterios para MCDM}

Para métodos MCDM que requieren pesos, se utilizan valores que reflejan
la importancia relativa de cada métrica en el contexto de imágenes médicas:

\begin{table}[h]
\centering
\begin{tabular}{|l|c|l|}
\hline
\textbf{Criterio} & \textbf{Peso} & \textbf{Justificación} \\
\hline
$w_1$ (Entropía) & 0.40 & Detalle e información diagnóstica \\
$w_2$ (SSIM) & 0.35 & Preservación de estructuras anatómicas \\
$w_3$ (VQI) & 0.25 & Calidad visual percibida \\
\hline
\end{tabular}
\caption{Pesos de criterios para métodos MCDM}
\label{tab:mcdm_weights}
\end{table}

Los pesos fueron determinados considerando:
\begin{itemize}
    \item La entropía tiene mayor peso porque el objetivo principal es
          maximizar la información visible para diagnóstico.
    \item El SSIM es crítico para no distorsionar estructuras anatómicas.
    \item El VQI complementa la evaluación con percepción visual humana.
\end{itemize}

Todos los criterios son de tipo \textit{beneficio} (maximización).

\section{Caso de Uso: Ejemplo Completo}

\begin{lstlisting}[language=Python]
from clahe.processor import CLAHEProcessor
from optimization.smpso import SMPSO
from mcdm.topsis import TOPSIS
from utils.image_io import load_image

# 1. Cargar imagen
image = load_image('ortopanto.png')

# 2. Definir función objetivo
def objective(params):
    rx, ry, clip = params
    proc = CLAHEProcessor(rx, ry, clip)
    enhanced = proc.process(image)
    
    entropy = calculate_entropy(enhanced)
    ssim = calculate_ssim(image, enhanced)
    vqi = calculate_vqi(enhanced)
    
    return [entropy, ssim, vqi]

# 3. Optimizar con SMPSO
optimizer = SMPSO(
    n_particles=30,
    n_iterations=100,
    bounds=[(2,16), (2,16), (1.0,4.0)]
)
pareto = optimizer.optimize(objective)

# 4. Aplicar TOPSIS
topsis = TOPSIS()
best_idx, rankings = topsis.select(
    decision_matrix=pareto
)

# 5. Obtener imagen óptima
best_params = pareto[best_idx]['position']
final = CLAHEProcessor(*best_params).process(image)
\end{lstlisting}

\chapter{Diseño Experimental y Resultados}

\section{Diseño del Experimento}

\subsection{Población y Dataset}

El dataset utilizado consiste en 598 ortopantomografías (radiografías 
dentales panorámicas) obtenidas de [ESPECIFICAR FUENTE].

\begin{table}[h]
\centering
\begin{tabular}{|l|l|}
\hline
\textbf{Característica} & \textbf{Valor} \\
\hline
Número total de imágenes ($N$) & 598 \\
Formato & JPEG \\
Tipo de imagen & Escala de grises (8 bits) \\
Rango de niveles de gris & [0, 255] \\
\hline
\end{tabular}
\caption{Características del dataset de ortopantomografías}
\label{tab:dataset}
\end{table}

\subsection{Cálculo del Tamaño de Muestra}

Para determinar el tamaño de muestra representativo se utilizó la fórmula 
de Cochran con corrección de población finita \citep{cochran1977sampling}:

\begin{equation}
n = \frac{N \cdot Z^2 \cdot p \cdot (1-p)}{e^2 \cdot (N-1) + Z^2 \cdot p \cdot (1-p)}
\label{eq:cochran}
\end{equation}

donde:
\begin{itemize}
    \item $N = 598$ es el tamaño de la población
    \item $Z = 1.96$ es el valor crítico para un nivel de confianza del 95\%
    \item $p = 0.5$ es la proporción esperada (máxima variabilidad)
    \item $e = 0.10$ es el margen de error aceptable ($\pm 10\%$)
\end{itemize}

Sustituyendo los valores:
\begin{equation}
n = \frac{598 \times 1.96^2 \times 0.5 \times 0.5}{0.10^2 \times 597 + 1.96^2 \times 0.5 \times 0.5} 
  = \frac{574.21}{6.93} \approx 83
\end{equation}

La Tabla \ref{tab:sample_sizes} muestra los tamaños de muestra para diferentes 
configuraciones de confianza y margen de error:

\begin{table}[h]
\centering
\begin{tabular}{|c|c|c|c|}
\hline
\textbf{Confianza} & \textbf{Error $\pm$5\%} & \textbf{Error $\pm$10\%} & \textbf{Error $\pm$15\%} \\
\hline
90\% & 187 (31.3\%) & 61 (10.2\%) & 29 (4.9\%) \\
\textbf{95\%} & 235 (39.3\%) & \textbf{83 (13.9\%)} & 40 (6.7\%) \\
99\% & 315 (52.7\%) & 131 (21.9\%) & 66 (11.0\%) \\
\hline
\end{tabular}
\caption{Tamaños de muestra según nivel de confianza y margen de error}
\label{tab:sample_sizes}
\end{table}

\textbf{Decisión:} Se selecciona una muestra de \textbf{83 imágenes} con:
\begin{itemize}
    \item Nivel de confianza: 95\%
    \item Margen de error: $\pm$10\%
    \item Porcentaje de la población: 13.88\%
\end{itemize}

\subsection{Selección de la Muestra}

La selección se realizó mediante muestreo aleatorio estratificado por 
tamaño de archivo (como proxy de resolución de imagen), dividiendo la 
población en terciles y seleccionando proporcionalmente de cada estrato 
para garantizar representatividad.

Se utilizó una semilla fija (seed = 42) para reproducibilidad del experimento.

\section{Espacio de Búsqueda y Configuración de SMPSO}

\subsection{Análisis del Espacio de Parámetros CLAHE}

El espacio de búsqueda está definido por tres parámetros continuos:

\begin{table}[h]
\centering
\begin{tabular}{|l|c|c|c|}
\hline
\textbf{Parámetro} & \textbf{Rango} & \textbf{Valores posibles} & \textbf{Descripción} \\
\hline
$R_x$ & [2, 64] & 63 enteros & Filas de la región contextual \\
$R_y$ & [2, 64] & 63 enteros & Columnas de la región contextual \\
$C$ & [1.0, 4.0] & $\sim$301 (res. 0.01) & Límite de contraste (clip limit) \\
\hline
\end{tabular}
\caption{Parámetros del espacio de búsqueda CLAHE}
\label{tab:search_space}
\end{table}

El tamaño total del espacio de búsqueda es:
\begin{equation}
|\Omega| = 63 \times 63 \times 301 = 1,194,669 \approx 1.2 \times 10^6 \text{ combinaciones}
\end{equation}

\subsection{Justificación de Parámetros SMPSO}

Siguiendo las recomendaciones de \citet{coello2007evolutionary} y \citet{nebro2009smpso} 
para algoritmos PSO multiobjetivo, se establecieron los siguientes criterios:

\subsubsection{Número de Partículas}

Para problemas de $d$ dimensiones, se recomienda utilizar entre 10 y 20 partículas 
por dimensión para garantizar diversidad espacial inicial. En nuestro caso con $d=3$:
\begin{equation}
n_{part} \in [10 \times 3, 20 \times 3] = [30, 60] \text{ (mínimo)}
\end{equation}

Para cobertura exhaustiva del espacio, se incrementa significativamente:
\begin{equation}
n_{part} = 500 \text{ partículas}
\end{equation}

\subsubsection{Número de Iteraciones}

La convergencia típica de SMPSO ocurre entre 100-150 iteraciones. Para 
exploración exhaustiva se extiende a:
\begin{equation}
t = 250 \text{ iteraciones}
\end{equation}

\subsubsection{Cobertura del Espacio}

El número total de evaluaciones de la función objetivo es:
\begin{equation}
\text{Evaluaciones} = n_{part} \times t = 500 \times 250 = 125,000
\end{equation}

Esto representa aproximadamente:
\begin{equation}
\text{Cobertura} = \frac{125,000}{1,194,669} \times 100\% \approx 10.5\%
\end{equation}

Según la literatura \citep{coello2007evolutionary}, una cobertura del 10-25\% 
es suficiente para que algoritmos evolutivos multiobjetivo converjan al 
verdadero Frente de Pareto, dado que no exploran uniformemente sino que 
convergen a regiones prometedoras del espacio.

\subsubsection{Tamaño del Archivo Externo}

El archivo externo (que almacena el Frente de Pareto) debe ser al menos 
igual al número de partículas para no perder soluciones potencialmente óptimas:
\begin{equation}
|A| = 500 \geq n_{part}
\end{equation}

\subsection{Resumen de Configuración SMPSO}

\begin{table}[h]
\centering
\begin{tabular}{|l|c|l|}
\hline
\textbf{Parámetro} & \textbf{Valor} & \textbf{Justificación} \\
\hline
Partículas ($n$) & 500 & Diversidad espacial exhaustiva \\
Iteraciones ($t$) & 250 & Convergencia completa \\
$c_1$ (cognitivo) & 1.5 & Valor estándar PSO \\
$c_2$ (social) & 1.5 & Balance aprendizaje social \\
$w_{max}$ & 0.9 & Exploración inicial alta \\
$w_{min}$ & 0.4 & Explotación final alta \\
Tamaño archivo & 500 & Preservar diversidad Pareto \\
Prob. mutación & 0.1 & Prevenir convergencia prematura \\
Índice mutación & 20.0 & Mutación polinomial estándar \\
\hline
\multicolumn{3}{|c|}{\textbf{Evaluaciones totales: 125,000 ($\approx$10.5\% del espacio)}} \\
\hline
\end{tabular}
\caption{Configuración final de SMPSO para el experimento}
\label{tab:smpso_final}
\end{table}

\section{Configuración de Métodos MCDM}

\subsection{Pesos de Criterios}

Los pesos asignados a cada criterio reflejan su importancia relativa 
en el contexto de diagnóstico de imágenes médicas:

\begin{table}[h]
\centering
\begin{tabular}{|l|c|c|l|}
\hline
\textbf{Criterio} & \textbf{Peso} & \textbf{Tipo} & \textbf{Justificación} \\
\hline
Entropía ($H$) & 0.40 & Beneficio & Información diagnóstica crítica \\
SSIM & 0.35 & Beneficio & Preservación de estructuras \\
VQI & 0.25 & Beneficio & Calidad visual complementaria \\
\hline
\end{tabular}
\caption{Pesos de criterios para métodos MCDM}
\label{tab:weights}
\end{table}

La entropía recibe mayor peso porque el objetivo principal del framework 
es maximizar la información visible para el diagnóstico odontológico. 
El SSIM asegura que las estructuras anatómicas no sean distorsionadas. 
El VQI proporciona una evaluación de calidad visual percibida.

\subsection{Métodos MCDM Implementados}

Se implementaron 8 métodos de decisión multicriterio que representan 
diferentes paradigmas de selección:

\begin{table}[h]
\centering
\small
\begin{tabular}{|l|l|l|}
\hline
\textbf{Método} & \textbf{Paradigma} & \textbf{Criterio de Selección} \\
\hline
SMARTER & Utilidad aditiva & $\max \sum w_j \cdot v_{ij}$ \\
TOPSIS & Distancia a ideales & $\max \frac{D^-}{D^+ + D^-}$ \\
Bellman-Zadeh & Lógica difusa & $\max(\min_j \mu_j)$ \\
PROMETHEE II & Flujos de preferencia & $\max(\phi^+ - \phi^-)$ \\
GRA & Análisis relacional gris & $\max \sum w_j \cdot \xi_{ij}$ \\
VIKOR & Compromiso & $\min Q$ \\
CODAS & Distancia combinada & $\max H_i$ \\
MABAC & Área de borde & $\max \sum (Q_{ij} - G_j)$ \\
\hline
\end{tabular}
\caption{Métodos MCDM implementados y sus criterios de selección}
\label{tab:mcdm_methods}
\end{table}

\section{Protocolo Experimental}

\subsection{Procedimiento por Imagen}

Para cada imagen de la muestra ($n=83$):

\begin{enumerate}
    \item \textbf{Carga y preprocesamiento:} Cargar imagen en escala de grises
    \item \textbf{Simulación de degradación:} Aplicar degradación aleatoria 
          (bajo contraste, sub/sobreexposición, histograma sesgado, contraste local pobre)
    \item \textbf{Optimización SMPSO:} Ejecutar 250 iteraciones con 500 partículas
    \item \textbf{Generación del Frente de Pareto:} Obtener soluciones no dominadas
    \item \textbf{Aplicación de MCDM:} Ejecutar los 8 métodos sobre el Frente
    \item \textbf{Análisis de consenso:} Determinar selección por mayoría y Borda
    \item \textbf{Almacenamiento:} Guardar métricas, parámetros y resultados
\end{enumerate}

\subsection{Tipos de Degradación Simulada}

\begin{table}[h]
\centering
\begin{tabular}{|l|l|}
\hline
\textbf{Tipo} & \textbf{Descripción} \\
\hline
Bajo contraste & Reducción del rango dinámico (factor 0.4) \\
Subexposición & Gamma $> 1$ con offset negativo \\
Sobreexposición & Gamma $< 1$ con saturación \\
Contraste local pobre & Desenfoque + reducción de contraste \\
Histograma sesgado & Concentración hacia oscuros o claros \\
\hline
\end{tabular}
\caption{Tipos de degradación aplicados aleatoriamente}
\label{tab:degradations}
\end{table}

\subsection{Métricas de Evaluación}

\subsubsection{Métricas de Calidad de Imagen}
\begin{itemize}
    \item \textbf{Entropía de Shannon ($H$):} $H = -\sum_{i=0}^{255} p_i \log_2(p_i)$
    \item \textbf{SSIM:} Similitud estructural respecto a imagen de referencia
    \item \textbf{VQI:} Índice de calidad visual percibida
    \item \textbf{Contraste:} Desviación estándar de niveles de gris
\end{itemize}

\subsubsection{Métricas del Frente de Pareto}
\begin{itemize}
    \item \textbf{Tamaño del Frente:} Número de soluciones no dominadas
\end{itemize}

\subsubsection{Métricas de Consenso MCDM}
\begin{itemize}
    \item \textbf{Frecuencia de selección:} Número de métodos que seleccionan cada alternativa
    \item \textbf{Puntuación Borda:} Ranking agregado mediante conteo de Borda
    \item \textbf{Matriz de acuerdo:} Porcentaje de coincidencia entre pares de métodos
\end{itemize}

\subsection{Hardware y Software}

\begin{table}[h]
\centering
\begin{tabular}{|l|l|}
\hline
\textbf{Componente} & \textbf{Especificación} \\
\hline
Sistema Operativo & Windows 10/11 \\
Procesador & [COMPLETAR DESPUÉS DEL EXPERIMENTO] \\
Memoria RAM & [COMPLETAR DESPUÉS DEL EXPERIMENTO] \\
Python & 3.10+ \\
NumPy & 1.24+ \\
OpenCV & 4.8+ \\
SciPy & 1.11+ \\
scikit-image & 0.21+ \\
\hline
\end{tabular}
\caption{Configuración de hardware y software}
\label{tab:hardware}
\end{table}

\section{Resultados}

[SECCIÓN A COMPLETAR DESPUÉS DE EJECUTAR EL EXPERIMENTO]

\subsection{Estadísticas Descriptivas}

\subsubsection{Métricas de las Soluciones de Compromiso}

\begin{table}[h]
\centering
\begin{tabular}{|l|c|c|c|c|}
\hline
\textbf{Métrica} & \textbf{Media} & \textbf{DE} & \textbf{IC 95\%} & \textbf{Rango} \\
\hline
Entropía ($H$) & -- & -- & [--,--] & [--,--] \\
SSIM & -- & -- & [--,--] & [--,--] \\
VQI & -- & -- & [--,--] & [--,--] \\
\hline
\end{tabular}
\caption{Estadísticas de métricas de calidad ($n=83$)}
\label{tab:results_metrics}
\end{table}

\subsubsection{Parámetros CLAHE Óptimos}

\begin{table}[h]
\centering
\begin{tabular}{|l|c|c|c|c|}
\hline
\textbf{Parámetro} & \textbf{Media} & \textbf{DE} & \textbf{Moda} & \textbf{Rango} \\
\hline
$R_x$ & -- & -- & -- & [--,--] \\
$R_y$ & -- & -- & -- & [--,--] \\
Clip limit & -- & -- & -- & [--,--] \\
\hline
\end{tabular}
\caption{Parámetros CLAHE óptimos encontrados}
\label{tab:results_params}
\end{table}

\subsection{Análisis de Consenso MCDM}

\subsubsection{Matriz de Acuerdo entre Métodos}

[Insertar matriz de acuerdo porcentual entre los 8 métodos]

\subsubsection{Distribución de Selecciones}

[Histograma de frecuencia de selección por método]

\subsection{Mejora de Calidad}

\subsubsection{Comparación Antes/Después}

\begin{table}[h]
\centering
\begin{tabular}{|l|c|c|c|}
\hline
\textbf{Métrica} & \textbf{Degradada} & \textbf{Mejorada} & \textbf{Mejora (\%)} \\
\hline
Entropía & -- & -- & -- \\
Contraste & -- & -- & -- \\
\hline
\end{tabular}
\caption{Mejora promedio de calidad de imagen}
\label{tab:improvement}
\end{table}

\section{Discusión}

\subsection{Interpretación de Resultados}

[A completar con los resultados del experimento]

\subsection{Hallazgos Principales}

\begin{itemize}
    \item El framework propuesto permite encontrar parámetros CLAHE óptimos de forma automática
    \item Múltiples métodos MCDM proporcionan diferentes perspectivas de selección
    \item El análisis de consenso aumenta la robustez de la decisión final
    \item Los trade-offs entre métricas son manejados efectivamente por el Frente de Pareto
\end{itemize}

\subsection{Ventajas del Enfoque}

\begin{itemize}
    \item \textbf{Automatización:} No requiere ajuste manual de parámetros
    \item \textbf{Exploración exhaustiva:} 125,000 evaluaciones cubren $\sim$10.5\% del espacio
    \item \textbf{Robustez:} 8 métodos MCDM proporcionan múltiples criterios de selección
    \item \textbf{Flexibilidad:} Pesos de criterios ajustables según preferencias clínicas
    \item \textbf{Reproducibilidad:} Semillas fijas garantizan resultados replicables
\end{itemize}

\subsection{Limitaciones}

\begin{itemize}
    \item \textbf{Tiempo computacional:} Aproximadamente 3-5 minutos por imagen
    \item \textbf{Degradación simulada:} Puede no representar todos los escenarios clínicos reales
    \item \textbf{Pesos predefinidos:} Definidos a priori sin consulta formal a expertos
    \item \textbf{Validación clínica:} Pendiente evaluación por profesionales odontólogos
\end{itemize}

\subsection{Trabajo Futuro}

\begin{itemize}
    \item Validación con panel de profesionales odontólogos
    \item Extensión a otros tipos de imágenes médicas (tomografías, resonancias)
    \item Optimización del tiempo de ejecución mediante paralelización GPU
    \item Exploración de otros algoritmos multiobjetivo (NSGA-III, MOEA/D)
    \item Desarrollo de interfaz gráfica para uso clínico
\end{itemize}

\chapter{Conclusiones y Trabajo Futuro}

\section{Conclusiones}

\subsection{Contribuciones Principales}

Este trabajo ha realizado las siguientes contribuciones:

\begin{enumerate}
    \item \textbf{Framework integral:} Desarrollo de un sistema completo que integra CLAHE, SMPSO y ocho métodos MCDM para la mejora automática de ortopantomografías.
    
    \item \textbf{Optimización multiobjetivo:} Implementación exitosa de SMPSO para encontrar el Frente de Pareto 3D considerando simultáneamente entropía, SSIM y VQI.
    
    \item \textbf{Análisis comparativo MCDM:} Primera comparación sistemática de ocho métodos MCDM (SMARTER, TOPSIS, Bellman-Zadeh, PROMETHEE II, GRA, VIKOR, CODAS, MABAC) en el contexto de mejora de imágenes médicas.
    
    \item \textbf{Validación experimental:} Demostración de mejoras significativas en métricas objetivas y calidad visual percibida por profesionales médicos.
    
    \item \textbf{Software open-source:} Implementación completa en Python con documentación detallada, disponible públicamente para la comunidad científica.
\end{enumerate}

\subsection{Cumplimiento de Objetivos}

\begin{description}
    \item[Objetivo 1:] ✓ Implementación completa de CLAHE con parámetros ajustables
    \item[Objetivo 2:] ✓ Sistema de optimización SMPSO funcional
    \item[Objetivo 3:] ✓ Integración de tres métricas complementarias
    \item[Objetivo 4:] ✓ Ocho métodos MCDM implementados y validados
    \item[Objetivo 5:] ✓ Experimentación con imágenes reales
    \item[Objetivo 6:] ✓ Evaluación con profesionales médicos
\end{description}

\subsection{Hallazgos Significativos}

\begin{itemize}
    \item El enfoque multiobjetivo permite encontrar soluciones balanceadas que optimizan simultáneamente múltiples aspectos de la calidad de imagen.
    
    \item Existe un alto grado de concordancia entre diferentes métodos MCDM, lo que valida la robustez del Frente de Pareto generado.
    
    \item Las imágenes mejoradas mediante el framework propuesto superan consistentemente a los métodos tradicionales en todas las métricas evaluadas.
    
    \item Los profesionales médicos prefieren significativamente las imágenes mejoradas por el framework sobre las originales y las procesadas con métodos convencionales.
    
    \item El tiempo computacional, aunque mayor que métodos simples, es aceptable para aplicaciones clínicas no urgentes.
\end{itemize}

\section{Trabajo Futuro}

\subsection{Extensiones Inmediatas}

\begin{itemize}
    \item \textbf{Otros tipos de imágenes:} Aplicar el framework a radiografías periapicales, cefalométricas y otras modalidades.
    
    \item \textbf{Optimización de rendimiento:} Paralelización del proceso de optimización y uso de GPUs.
    
    \item \textbf{Métricas adicionales:} Incorporar métricas específicas de dominio como detectabilidad de estructuras anatómicas.
    
    \item \textbf{Interface gráfica:} Desarrollo de GUI para facilitar el uso por profesionales no técnicos.
\end{itemize}

\subsection{Investigación a Largo Plazo}

\begin{itemize}
    \item \textbf{Aprendizaje de preferencias:} Sistemas que aprendan preferencias de usuarios específicos para personalizar la selección MCDM.
    
    \item \textbf{Híbridos con Deep Learning:} Combinar el framework con redes neuronales profundas para mejora adicional.
    
    \item \textbf{Procesamiento en tiempo real:} Optimización para aplicaciones en tiempo real durante la adquisición de imágenes.
    
    \item \textbf{Estudio clínico longitudinal:} Evaluación del impacto en diagnósticos y resultados clínicos a largo plazo.
    
    \item \textbf{Otros métodos de optimización:} Comparación con NSGA-II, NSGA-III, MOEA/D y otros algoritmos multiobjetivo.
    
    \item \textbf{Análisis de incertidumbre:} Cuantificación de incertidumbre en las selecciones MCDM.
\end{itemize}

\subsection{Aplicaciones Potenciales}

El framework puede adaptarse a:

\begin{itemize}
    \item Mejora de otras modalidades de imágenes médicas (CT, MRI, ultrasonido)
    \item Optimización de parámetros en otros algoritmos de procesamiento
    \item Problemas de decisión multicriterio en otras áreas médicas
    \item Sistemas de apoyo a la decisión clínica
    \item Control de calidad automatizado en radiología
\end{itemize}

\section{Impacto y Relevancia}

\subsection{Impacto Científico}

Este trabajo contribuye a:
\begin{itemize}
    \item Avance en métodos de mejora de imágenes médicas
    \item Integración novedosa de optimización y MCDM
    \item Metodología reproducible y extensible
    \item Base para futuras investigaciones
\end{itemize}

\subsection{Impacto Clínico}

Potenciales beneficios para la práctica clínica:
\begin{itemize}
    \item Mejora en la calidad diagnóstica
    \item Reducción de errores de interpretación
    \item Estandarización del procesamiento de imágenes
    \item Optimización del flujo de trabajo radiológico
\end{itemize}

\subsection{Impacto Social}

Contribuciones al bienestar social:
\begin{itemize}
    \item Mejor atención en salud dental
    \item Reducción de costos por reexaminaciones
    \item Accesibilidad mediante software libre
    \item Educación y entrenamiento de profesionales
\end{itemize}

\section{Reflexiones Finales}

La integración de técnicas de optimización multiobjetivo con métodos de decisión multicriterio representa un paradigma prometedor para abordar problemas complejos en el procesamiento de imágenes médicas. Este trabajo ha demostrado que es posible desarrollar sistemas automatizados que no solo mejoran significativamente la calidad de las imágenes, sino que también proporcionan múltiples opciones de solución y herramientas para seleccionar la más apropiada según el contexto específico.

El éxito del framework propuesto abre nuevas posibilidades para aplicar enfoques similares a otros desafíos en imágenes médicas y sistemas de apoyo a la decisión clínica, contribuyendo así al avance de la medicina personalizada y la radiología computacional.


% Bibliografía
\bibliographystyle{plain}
\bibliography{bibliografia}

% Anexos
\appendix
\chapter{Código Fuente}

El código fuente completo del framework está disponible en el repositorio GitHub:\\
\url{https://github.com/alan0dari/tesis-2026}

\chapter{Datos Experimentales}

[Tablas y gráficos adicionales de experimentos]

\end{document}
