\chapter{Metodología Propuesta}

\section{Arquitectura del Framework}

El framework propuesto consta de cinco módulos principales:

\begin{enumerate}
    \item \textbf{Módulo de Procesamiento CLAHE}
    \item \textbf{Módulo de Métricas de Evaluación}
    \item \textbf{Módulo de Optimización Multiobjetivo}
    \item \textbf{Módulo de Métodos MCDM}
    \item \textbf{Módulo de Visualización}
\end{enumerate}

\section{Flujo de Trabajo}

\begin{algorithm}
\caption{Framework CLAHE-MCDM}
\begin{algorithmic}
\STATE \textbf{Entrada:} Ortopantomografía $I$
\STATE \textbf{Salida:} Imagen mejorada óptima $I^*$
\STATE
\STATE // Fase 1: Optimización
\STATE Definir función objetivo $\mathbf{F}(R_x, R_y, C) = [Entropía, SSIM, VQI]$
\STATE Ejecutar SMPSO con $n$ partículas, $t$ iteraciones
\STATE Obtener Frente de Pareto $\mathcal{P}$
\STATE
\STATE // Fase 2: Selección MCDM
\STATE Construir matriz de decisión desde $\mathcal{P}$
\FOR{cada método MCDM $m \in \{$SMARTER, TOPSIS, ..., MABAC$\}$}
    \STATE Aplicar método $m$
    \STATE Obtener ranking y mejor solución $s_m$
\ENDFOR
\STATE
\STATE // Fase 3: Consenso
\STATE Analizar concordancia entre métodos
\STATE Seleccionar solución final $I^*$
\STATE \textbf{Retornar} $I^*$
\end{algorithmic}
\end{algorithm}

\section{Implementación}

\subsection{Tecnologías Utilizadas}

\begin{itemize}
    \item \textbf{Lenguaje:} Python 3.8+
    \item \textbf{Librerías principales:}
    \begin{itemize}
        \item NumPy, SciPy: Computación científica
        \item OpenCV: Procesamiento de imágenes
        \item scikit-image: Métricas de calidad
        \item Matplotlib: Visualización
        \item pandas: Manejo de datos
    \end{itemize}
\end{itemize}

\subsection{Estructura del Código}

\begin{lstlisting}[language=bash]
src/
├── clahe/          # Procesamiento CLAHE
├── metrics/        # Métricas (entropía, SSIM, VQI)
├── optimization/   # SMPSO y Pareto
├── mcdm/          # 8 métodos MCDM
└── utils/         # Utilidades
\end{lstlisting}

\section{Configuración Experimental}

\subsection{Parámetros de CLAHE}

\begin{table}[h]
\centering
\begin{tabular}{|l|l|l|}
\hline
\textbf{Parámetro} & \textbf{Rango} & \textbf{Descripción} \\
\hline
$R_x$ & [2, 16] & Regiones horizontales \\
$R_y$ & [2, 16] & Regiones verticales \\
$C$ & [1.0, 4.0] & Límite de contraste \\
\hline
\end{tabular}
\caption{Rangos de parámetros CLAHE}
\end{table}

\subsection{Parámetros de SMPSO}

\begin{table}[h]
\centering
\begin{tabular}{|l|l|}
\hline
\textbf{Parámetro} & \textbf{Valor} \\
\hline
Partículas & 30 \\
Iteraciones & 100 \\
$c_1$ (cognitivo) & 1.5 \\
$c_2$ (social) & 1.5 \\
Tamaño archivo & 100 \\
Probabilidad mutación & 0.1 \\
\hline
\end{tabular}
\caption{Configuración SMPSO}
\end{table}

\subsection{Pesos de Criterios}

Para métodos MCDM que requieren pesos:
\begin{itemize}
    \item $w_1$ (Entropía) = 0.33
    \item $w_2$ (SSIM) = 0.33
    \item $w_3$ (VQI) = 0.34
\end{itemize}

\section{Caso de Uso: Ejemplo Completo}

\begin{lstlisting}[language=Python]
from clahe.processor import CLAHEProcessor
from optimization.smpso import SMPSO
from mcdm.topsis import TOPSIS
from utils.image_io import load_image

# 1. Cargar imagen
image = load_image('ortopanto.png')

# 2. Definir función objetivo
def objective(params):
    rx, ry, clip = params
    proc = CLAHEProcessor(rx, ry, clip)
    enhanced = proc.process(image)
    
    entropy = calculate_entropy(enhanced)
    ssim = calculate_ssim(image, enhanced)
    vqi = calculate_vqi(enhanced)
    
    return [entropy, ssim, vqi]

# 3. Optimizar con SMPSO
optimizer = SMPSO(
    n_particles=30,
    n_iterations=100,
    bounds=[(2,16), (2,16), (1.0,4.0)]
)
pareto = optimizer.optimize(objective)

# 4. Aplicar TOPSIS
topsis = TOPSIS()
best_idx, rankings = topsis.select(
    decision_matrix=pareto
)

# 5. Obtener imagen óptima
best_params = pareto[best_idx]['position']
final = CLAHEProcessor(*best_params).process(image)
\end{lstlisting}
