\chapter{Metodología Propuesta}

\section{Arquitectura del Framework}

El framework propuesto consta de cinco módulos principales:

\begin{enumerate}
    \item \textbf{Módulo de Procesamiento CLAHE}
    \item \textbf{Módulo de Métricas de Evaluación}
    \item \textbf{Módulo de Optimización Multiobjetivo}
    \item \textbf{Módulo de Métodos MCDM}
    \item \textbf{Módulo de Visualización}
\end{enumerate}

\section{Flujo de Trabajo}

\begin{algorithm}
\caption{Framework CLAHE-MCDM}
\begin{algorithmic}
\STATE \textbf{Entrada:} Ortopantomografía $I$
\STATE \textbf{Salida:} Imagen mejorada óptima $I^*$
\STATE
\STATE // Fase 1: Optimización
\STATE Definir función objetivo $\mathbf{F}(R_x, R_y, C) = [Entropía, SSIM, VQI]$
\STATE Ejecutar SMPSO con $n$ partículas, $t$ iteraciones
\STATE Obtener Frente de Pareto $\mathcal{P}$
\STATE
\STATE // Fase 2: Selección MCDM
\STATE Construir matriz de decisión desde $\mathcal{P}$
\FOR{cada método MCDM $m \in \{$SMARTER, TOPSIS, ..., MABAC$\}$}
    \STATE Aplicar método $m$
    \STATE Obtener ranking y mejor solución $s_m$
\ENDFOR
\STATE
\STATE // Fase 3: Consenso
\STATE Analizar concordancia entre métodos
\STATE Seleccionar solución final $I^*$
\STATE \textbf{Retornar} $I^*$
\end{algorithmic}
\end{algorithm}

\section{Implementación}

\subsection{Tecnologías Utilizadas}

\begin{itemize}
    \item \textbf{Lenguaje:} Python 3.8+
    \item \textbf{Librerías principales:}
    \begin{itemize}
        \item NumPy, SciPy: Computación científica
        \item OpenCV: Procesamiento de imágenes
        \item scikit-image: Métricas de calidad
        \item Matplotlib: Visualización
        \item pandas: Manejo de datos
    \end{itemize}
\end{itemize}

\subsection{Estructura del Código}

\begin{lstlisting}[language=bash]
src/
├── clahe/          # Procesamiento CLAHE
├── metrics/        # Métricas (entropía, SSIM, VQI)
├── optimization/   # SMPSO y Pareto
├── mcdm/          # 8 métodos MCDM
└── utils/         # Utilidades
\end{lstlisting}

\section{Configuración Experimental}

\subsection{Parámetros de CLAHE}

El espacio de búsqueda de CLAHE está definido por tres parámetros que determinan 
el comportamiento del algoritmo de mejora de contraste:

\begin{table}[h]
\centering
\begin{tabular}{|l|c|l|l|}
\hline
\textbf{Parámetro} & \textbf{Rango} & \textbf{Valores} & \textbf{Descripción} \\
\hline
$R_x$ & [2, 64] & 63 enteros & Regiones contextuales (filas) \\
$R_y$ & [2, 64] & 63 enteros & Regiones contextuales (columnas) \\
$C$ (clip limit) & [1.0, 4.0] & Continuo & Límite de contraste local \\
\hline
\end{tabular}
\caption{Rangos de parámetros CLAHE para optimización}
\label{tab:clahe_params}
\end{table}

El espacio de búsqueda total comprende aproximadamente:
\begin{equation}
|\Omega| = 63 \times 63 \times 301 \approx 1,195,000 \text{ combinaciones}
\end{equation}

donde 301 representa la discretización del clip limit con resolución de 0.01.

\subsection{Parámetros de SMPSO}

La configuración del algoritmo SMPSO se seleccionó para garantizar una exploración
exhaustiva del espacio de búsqueda, siguiendo las recomendaciones de 
\citet{coello2007evolutionary} para optimización multiobjetivo:

\begin{table}[h]
\centering
\begin{tabular}{|l|c|l|}
\hline
\textbf{Parámetro} & \textbf{Valor} & \textbf{Justificación} \\
\hline
Partículas ($n$) & 500 & $\approx 10\text{-}20 \times$ dimensiones \\
Iteraciones ($t$) & 250 & Convergencia completa \\
$c_1$ (cognitivo) & 1.5 & Balance exploración/explotación \\
$c_2$ (social) & 1.5 & Valor estándar \\
$w_{max}$ (inercia inicial) & 0.9 & Mayor exploración inicial \\
$w_{min}$ (inercia final) & 0.4 & Mayor explotación final \\
Tamaño archivo & 500 & $\geq n$ para preservar diversidad \\
Prob. mutación & 0.1 & Mantener diversidad \\
\hline
\end{tabular}
\caption{Configuración SMPSO para cobertura exhaustiva}
\label{tab:smpso_config}
\end{table}

Con esta configuración, el número total de evaluaciones de la función objetivo es:
\begin{equation}
\text{Evaluaciones} = n \times t = 500 \times 250 = 125,000
\end{equation}

Esto representa aproximadamente el 10.5\% del espacio de búsqueda total,
lo cual es suficiente para que SMPSO converja al verdadero Frente de Pareto 
\citep{nebro2009smpso}.

\subsection{Pesos de Criterios para MCDM}

Para métodos MCDM que requieren pesos, se utilizan valores que reflejan
la importancia relativa de cada métrica en el contexto de imágenes médicas:

\begin{table}[h]
\centering
\begin{tabular}{|l|c|l|}
\hline
\textbf{Criterio} & \textbf{Peso} & \textbf{Justificación} \\
\hline
$w_1$ (Entropía) & 0.40 & Detalle e información diagnóstica \\
$w_2$ (SSIM) & 0.35 & Preservación de estructuras anatómicas \\
$w_3$ (VQI) & 0.25 & Calidad visual percibida \\
\hline
\end{tabular}
\caption{Pesos de criterios para métodos MCDM}
\label{tab:mcdm_weights}
\end{table}

Los pesos fueron determinados considerando:
\begin{itemize}
    \item La entropía tiene mayor peso porque el objetivo principal es
          maximizar la información visible para diagnóstico.
    \item El SSIM es crítico para no distorsionar estructuras anatómicas.
    \item El VQI complementa la evaluación con percepción visual humana.
\end{itemize}

Todos los criterios son de tipo \textit{beneficio} (maximización).

\section{Caso de Uso: Ejemplo Completo}

\begin{lstlisting}[language=Python]
from clahe.processor import CLAHEProcessor
from optimization.smpso import SMPSO
from mcdm.topsis import TOPSIS
from utils.image_io import load_image

# 1. Cargar imagen
image = load_image('ortopanto.png')

# 2. Definir función objetivo
def objective(params):
    rx, ry, clip = params
    proc = CLAHEProcessor(rx, ry, clip)
    enhanced = proc.process(image)
    
    entropy = calculate_entropy(enhanced)
    ssim = calculate_ssim(image, enhanced)
    vqi = calculate_vqi(enhanced)
    
    return [entropy, ssim, vqi]

# 3. Optimizar con SMPSO
optimizer = SMPSO(
    n_particles=30,
    n_iterations=100,
    bounds=[(2,16), (2,16), (1.0,4.0)]
)
pareto = optimizer.optimize(objective)

# 4. Aplicar TOPSIS
topsis = TOPSIS()
best_idx, rankings = topsis.select(
    decision_matrix=pareto
)

# 5. Obtener imagen óptima
best_params = pareto[best_idx]['position']
final = CLAHEProcessor(*best_params).process(image)
\end{lstlisting}
