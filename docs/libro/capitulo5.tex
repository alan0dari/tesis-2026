\chapter{Diseño Experimental y Resultados}

\section{Diseño del Experimento}

\subsection{Población y Dataset}

El dataset utilizado consiste en 598 ortopantomografías (radiografías 
dentales panorámicas) obtenidas de [ESPECIFICAR FUENTE].

\begin{table}[h]
\centering
\begin{tabular}{|l|l|}
\hline
\textbf{Característica} & \textbf{Valor} \\
\hline
Número total de imágenes ($N$) & 598 \\
Formato & JPEG \\
Tipo de imagen & Escala de grises (8 bits) \\
Rango de niveles de gris & [0, 255] \\
\hline
\end{tabular}
\caption{Características del dataset de ortopantomografías}
\label{tab:dataset}
\end{table}

\subsection{Cálculo del Tamaño de Muestra}

Para determinar el tamaño de muestra representativo se utilizó la fórmula 
de Cochran con corrección de población finita \citep{cochran1977sampling}:

\begin{equation}
n = \frac{N \cdot Z^2 \cdot p \cdot (1-p)}{e^2 \cdot (N-1) + Z^2 \cdot p \cdot (1-p)}
\label{eq:cochran}
\end{equation}

donde:
\begin{itemize}
    \item $N = 598$ es el tamaño de la población
    \item $Z = 1.96$ es el valor crítico para un nivel de confianza del 95\%
    \item $p = 0.5$ es la proporción esperada (máxima variabilidad)
    \item $e = 0.10$ es el margen de error aceptable ($\pm 10\%$)
\end{itemize}

Sustituyendo los valores:
\begin{equation}
n = \frac{598 \times 1.96^2 \times 0.5 \times 0.5}{0.10^2 \times 597 + 1.96^2 \times 0.5 \times 0.5} 
  = \frac{574.21}{6.93} \approx 83
\end{equation}

La Tabla \ref{tab:sample_sizes} muestra los tamaños de muestra para diferentes 
configuraciones de confianza y margen de error:

\begin{table}[h]
\centering
\begin{tabular}{|c|c|c|c|}
\hline
\textbf{Confianza} & \textbf{Error $\pm$5\%} & \textbf{Error $\pm$10\%} & \textbf{Error $\pm$15\%} \\
\hline
90\% & 187 (31.3\%) & 61 (10.2\%) & 29 (4.9\%) \\
\textbf{95\%} & 235 (39.3\%) & \textbf{83 (13.9\%)} & 40 (6.7\%) \\
99\% & 315 (52.7\%) & 131 (21.9\%) & 66 (11.0\%) \\
\hline
\end{tabular}
\caption{Tamaños de muestra según nivel de confianza y margen de error}
\label{tab:sample_sizes}
\end{table}

\textbf{Decisión:} Se selecciona una muestra de \textbf{83 imágenes} con:
\begin{itemize}
    \item Nivel de confianza: 95\%
    \item Margen de error: $\pm$10\%
    \item Porcentaje de la población: 13.88\%
\end{itemize}

\subsection{Selección de la Muestra}

La selección se realizó mediante muestreo aleatorio estratificado por 
tamaño de archivo (como proxy de resolución de imagen), dividiendo la 
población en terciles y seleccionando proporcionalmente de cada estrato 
para garantizar representatividad.

Se utilizó una semilla fija (seed = 42) para reproducibilidad del experimento.

\section{Espacio de Búsqueda y Configuración de SMPSO}

\subsection{Análisis del Espacio de Parámetros CLAHE}

El espacio de búsqueda está definido por tres parámetros continuos:

\begin{table}[h]
\centering
\begin{tabular}{|l|c|c|c|}
\hline
\textbf{Parámetro} & \textbf{Rango} & \textbf{Valores posibles} & \textbf{Descripción} \\
\hline
$R_x$ & [2, 64] & 63 enteros & Filas de la región contextual \\
$R_y$ & [2, 64] & 63 enteros & Columnas de la región contextual \\
$C$ & [1.0, 4.0] & $\sim$301 (res. 0.01) & Límite de contraste (clip limit) \\
\hline
\end{tabular}
\caption{Parámetros del espacio de búsqueda CLAHE}
\label{tab:search_space}
\end{table}

El tamaño total del espacio de búsqueda es:
\begin{equation}
|\Omega| = 63 \times 63 \times 301 = 1,194,669 \approx 1.2 \times 10^6 \text{ combinaciones}
\end{equation}

\subsection{Justificación de Parámetros SMPSO}

Siguiendo las recomendaciones de \citet{coello2007evolutionary} y \citet{nebro2009smpso} 
para algoritmos PSO multiobjetivo, se establecieron los siguientes criterios:

\subsubsection{Número de Partículas}

Para problemas de $d$ dimensiones, se recomienda utilizar entre 10 y 20 partículas 
por dimensión para garantizar diversidad espacial inicial. En nuestro caso con $d=3$:
\begin{equation}
n_{part} \in [10 \times 3, 20 \times 3] = [30, 60] \text{ (mínimo)}
\end{equation}

Para cobertura exhaustiva del espacio, se incrementa significativamente:
\begin{equation}
n_{part} = 500 \text{ partículas}
\end{equation}

\subsubsection{Número de Iteraciones}

La convergencia típica de SMPSO ocurre entre 100-150 iteraciones. Para 
exploración exhaustiva se extiende a:
\begin{equation}
t = 250 \text{ iteraciones}
\end{equation}

\subsubsection{Cobertura del Espacio}

El número total de evaluaciones de la función objetivo es:
\begin{equation}
\text{Evaluaciones} = n_{part} \times t = 500 \times 250 = 125,000
\end{equation}

Esto representa aproximadamente:
\begin{equation}
\text{Cobertura} = \frac{125,000}{1,194,669} \times 100\% \approx 10.5\%
\end{equation}

Según la literatura \citep{coello2007evolutionary}, una cobertura del 10-25\% 
es suficiente para que algoritmos evolutivos multiobjetivo converjan al 
verdadero Frente de Pareto, dado que no exploran uniformemente sino que 
convergen a regiones prometedoras del espacio.

\subsubsection{Tamaño del Archivo Externo}

El archivo externo (que almacena el Frente de Pareto) debe ser al menos 
igual al número de partículas para no perder soluciones potencialmente óptimas:
\begin{equation}
|A| = 500 \geq n_{part}
\end{equation}

\subsection{Resumen de Configuración SMPSO}

\begin{table}[h]
\centering
\begin{tabular}{|l|c|l|}
\hline
\textbf{Parámetro} & \textbf{Valor} & \textbf{Justificación} \\
\hline
Partículas ($n$) & 500 & Diversidad espacial exhaustiva \\
Iteraciones ($t$) & 250 & Convergencia completa \\
$c_1$ (cognitivo) & 1.5 & Valor estándar PSO \\
$c_2$ (social) & 1.5 & Balance aprendizaje social \\
$w_{max}$ & 0.9 & Exploración inicial alta \\
$w_{min}$ & 0.4 & Explotación final alta \\
Tamaño archivo & 500 & Preservar diversidad Pareto \\
Prob. mutación & 0.1 & Prevenir convergencia prematura \\
Índice mutación & 20.0 & Mutación polinomial estándar \\
\hline
\multicolumn{3}{|c|}{\textbf{Evaluaciones totales: 125,000 ($\approx$10.5\% del espacio)}} \\
\hline
\end{tabular}
\caption{Configuración final de SMPSO para el experimento}
\label{tab:smpso_final}
\end{table}

\section{Configuración de Métodos MCDM}

\subsection{Pesos de Criterios}

Los pesos asignados a cada criterio reflejan su importancia relativa 
en el contexto de diagnóstico de imágenes médicas:

\begin{table}[h]
\centering
\begin{tabular}{|l|c|c|l|}
\hline
\textbf{Criterio} & \textbf{Peso} & \textbf{Tipo} & \textbf{Justificación} \\
\hline
Entropía ($H$) & 0.40 & Beneficio & Información diagnóstica crítica \\
SSIM & 0.35 & Beneficio & Preservación de estructuras \\
VQI & 0.25 & Beneficio & Calidad visual complementaria \\
\hline
\end{tabular}
\caption{Pesos de criterios para métodos MCDM}
\label{tab:weights}
\end{table}

La entropía recibe mayor peso porque el objetivo principal del framework 
es maximizar la información visible para el diagnóstico odontológico. 
El SSIM asegura que las estructuras anatómicas no sean distorsionadas. 
El VQI proporciona una evaluación de calidad visual percibida.

\subsection{Métodos MCDM Implementados}

Se implementaron 8 métodos de decisión multicriterio que representan 
diferentes paradigmas de selección:

\begin{table}[h]
\centering
\small
\begin{tabular}{|l|l|l|}
\hline
\textbf{Método} & \textbf{Paradigma} & \textbf{Criterio de Selección} \\
\hline
SMARTER & Utilidad aditiva & $\max \sum w_j \cdot v_{ij}$ \\
TOPSIS & Distancia a ideales & $\max \frac{D^-}{D^+ + D^-}$ \\
Bellman-Zadeh & Lógica difusa & $\max(\min_j \mu_j)$ \\
PROMETHEE II & Flujos de preferencia & $\max(\phi^+ - \phi^-)$ \\
GRA & Análisis relacional gris & $\max \sum w_j \cdot \xi_{ij}$ \\
VIKOR & Compromiso & $\min Q$ \\
CODAS & Distancia combinada & $\max H_i$ \\
MABAC & Área de borde & $\max \sum (Q_{ij} - G_j)$ \\
\hline
\end{tabular}
\caption{Métodos MCDM implementados y sus criterios de selección}
\label{tab:mcdm_methods}
\end{table}

\section{Protocolo Experimental}

\subsection{Procedimiento por Imagen}

Para cada imagen de la muestra ($n=83$):

\begin{enumerate}
    \item \textbf{Carga y preprocesamiento:} Cargar imagen en escala de grises
    \item \textbf{Simulación de degradación:} Aplicar degradación aleatoria 
          (bajo contraste, sub/sobreexposición, histograma sesgado, contraste local pobre)
    \item \textbf{Optimización SMPSO:} Ejecutar 250 iteraciones con 500 partículas
    \item \textbf{Generación del Frente de Pareto:} Obtener soluciones no dominadas
    \item \textbf{Aplicación de MCDM:} Ejecutar los 8 métodos sobre el Frente
    \item \textbf{Análisis de consenso:} Determinar selección por mayoría y Borda
    \item \textbf{Almacenamiento:} Guardar métricas, parámetros y resultados
\end{enumerate}

\subsection{Tipos de Degradación Simulada}

\begin{table}[h]
\centering
\begin{tabular}{|l|l|}
\hline
\textbf{Tipo} & \textbf{Descripción} \\
\hline
Bajo contraste & Reducción del rango dinámico (factor 0.4) \\
Subexposición & Gamma $> 1$ con offset negativo \\
Sobreexposición & Gamma $< 1$ con saturación \\
Contraste local pobre & Desenfoque + reducción de contraste \\
Histograma sesgado & Concentración hacia oscuros o claros \\
\hline
\end{tabular}
\caption{Tipos de degradación aplicados aleatoriamente}
\label{tab:degradations}
\end{table}

\subsection{Métricas de Evaluación}

\subsubsection{Métricas de Calidad de Imagen}
\begin{itemize}
    \item \textbf{Entropía de Shannon ($H$):} $H = -\sum_{i=0}^{255} p_i \log_2(p_i)$
    \item \textbf{SSIM:} Similitud estructural respecto a imagen de referencia
    \item \textbf{VQI:} Índice de calidad visual percibida
    \item \textbf{Contraste:} Desviación estándar de niveles de gris
\end{itemize}

\subsubsection{Métricas del Frente de Pareto}
\begin{itemize}
    \item \textbf{Tamaño del Frente:} Número de soluciones no dominadas
\end{itemize}

\subsubsection{Métricas de Consenso MCDM}
\begin{itemize}
    \item \textbf{Frecuencia de selección:} Número de métodos que seleccionan cada alternativa
    \item \textbf{Puntuación Borda:} Ranking agregado mediante conteo de Borda
    \item \textbf{Matriz de acuerdo:} Porcentaje de coincidencia entre pares de métodos
\end{itemize}

\subsection{Hardware y Software}

\begin{table}[h]
\centering
\begin{tabular}{|l|l|}
\hline
\textbf{Componente} & \textbf{Especificación} \\
\hline
Sistema Operativo & Windows 10/11 \\
Procesador & [COMPLETAR DESPUÉS DEL EXPERIMENTO] \\
Memoria RAM & [COMPLETAR DESPUÉS DEL EXPERIMENTO] \\
Python & 3.10+ \\
NumPy & 1.24+ \\
OpenCV & 4.8+ \\
SciPy & 1.11+ \\
scikit-image & 0.21+ \\
\hline
\end{tabular}
\caption{Configuración de hardware y software}
\label{tab:hardware}
\end{table}

\section{Resultados}

[SECCIÓN A COMPLETAR DESPUÉS DE EJECUTAR EL EXPERIMENTO]

\subsection{Estadísticas Descriptivas}

\subsubsection{Métricas de las Soluciones de Compromiso}

\begin{table}[h]
\centering
\begin{tabular}{|l|c|c|c|c|}
\hline
\textbf{Métrica} & \textbf{Media} & \textbf{DE} & \textbf{IC 95\%} & \textbf{Rango} \\
\hline
Entropía ($H$) & -- & -- & [--,--] & [--,--] \\
SSIM & -- & -- & [--,--] & [--,--] \\
VQI & -- & -- & [--,--] & [--,--] \\
\hline
\end{tabular}
\caption{Estadísticas de métricas de calidad ($n=83$)}
\label{tab:results_metrics}
\end{table}

\subsubsection{Parámetros CLAHE Óptimos}

\begin{table}[h]
\centering
\begin{tabular}{|l|c|c|c|c|}
\hline
\textbf{Parámetro} & \textbf{Media} & \textbf{DE} & \textbf{Moda} & \textbf{Rango} \\
\hline
$R_x$ & -- & -- & -- & [--,--] \\
$R_y$ & -- & -- & -- & [--,--] \\
Clip limit & -- & -- & -- & [--,--] \\
\hline
\end{tabular}
\caption{Parámetros CLAHE óptimos encontrados}
\label{tab:results_params}
\end{table}

\subsection{Análisis de Consenso MCDM}

\subsubsection{Matriz de Acuerdo entre Métodos}

[Insertar matriz de acuerdo porcentual entre los 8 métodos]

\subsubsection{Distribución de Selecciones}

[Histograma de frecuencia de selección por método]

\subsection{Mejora de Calidad}

\subsubsection{Comparación Antes/Después}

\begin{table}[h]
\centering
\begin{tabular}{|l|c|c|c|}
\hline
\textbf{Métrica} & \textbf{Degradada} & \textbf{Mejorada} & \textbf{Mejora (\%)} \\
\hline
Entropía & -- & -- & -- \\
Contraste & -- & -- & -- \\
\hline
\end{tabular}
\caption{Mejora promedio de calidad de imagen}
\label{tab:improvement}
\end{table}

\section{Discusión}

\subsection{Interpretación de Resultados}

[A completar con los resultados del experimento]

\subsection{Hallazgos Principales}

\begin{itemize}
    \item El framework propuesto permite encontrar parámetros CLAHE óptimos de forma automática
    \item Múltiples métodos MCDM proporcionan diferentes perspectivas de selección
    \item El análisis de consenso aumenta la robustez de la decisión final
    \item Los trade-offs entre métricas son manejados efectivamente por el Frente de Pareto
\end{itemize}

\subsection{Ventajas del Enfoque}

\begin{itemize}
    \item \textbf{Automatización:} No requiere ajuste manual de parámetros
    \item \textbf{Exploración exhaustiva:} 125,000 evaluaciones cubren $\sim$10.5\% del espacio
    \item \textbf{Robustez:} 8 métodos MCDM proporcionan múltiples criterios de selección
    \item \textbf{Flexibilidad:} Pesos de criterios ajustables según preferencias clínicas
    \item \textbf{Reproducibilidad:} Semillas fijas garantizan resultados replicables
\end{itemize}

\subsection{Limitaciones}

\begin{itemize}
    \item \textbf{Tiempo computacional:} Aproximadamente 3-5 minutos por imagen
    \item \textbf{Degradación simulada:} Puede no representar todos los escenarios clínicos reales
    \item \textbf{Pesos predefinidos:} Definidos a priori sin consulta formal a expertos
    \item \textbf{Validación clínica:} Pendiente evaluación por profesionales odontólogos
\end{itemize}

\subsection{Trabajo Futuro}

\begin{itemize}
    \item Validación con panel de profesionales odontólogos
    \item Extensión a otros tipos de imágenes médicas (tomografías, resonancias)
    \item Optimización del tiempo de ejecución mediante paralelización GPU
    \item Exploración de otros algoritmos multiobjetivo (NSGA-III, MOEA/D)
    \item Desarrollo de interfaz gráfica para uso clínico
\end{itemize}
