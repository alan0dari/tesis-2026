\chapter{Marco Teórico: Imágenes Médicas y Ortopantomografías}

\section{Imágenes Médicas}

\subsection{Características de las Imágenes Médicas}

Las imágenes médicas son representaciones visuales de la anatomía interna del cuerpo humano obtenidas mediante diversas modalidades de adquisición. Se caracterizan por:

\begin{itemize}
    \item \textbf{Alta dimensionalidad:} Resoluciones de varios megapíxeles
    \item \textbf{Rango dinámico variable:} Dependiendo de la modalidad
    \item \textbf{Ruido inherente:} Debido al proceso de adquisición
    \item \textbf{Artefactos:} Distorsiones y anomalías del proceso de captura
\end{itemize}

\subsection{Modalidades de Imágenes Médicas}

\subsubsection{Rayos X}
Utilizan radiación ionizante para generar imágenes de estructuras óseas y tejidos densos.

\subsubsection{Tomografía Computarizada (CT)}
Genera imágenes tridimensionales mediante múltiples proyecciones de rayos X.

\subsubsection{Resonancia Magnética (MRI)}
Utiliza campos magnéticos y ondas de radio para visualizar tejidos blandos.

\subsubsection{Ultrasonido}
Emplea ondas sonoras de alta frecuencia para generar imágenes en tiempo real.

\section{Ortopantomografías}

\subsection{Definición y Principio de Funcionamiento}

La ortopantomografía, también conocida como radiografía panorámica dental, es una técnica radiológica que produce una imagen bidimensional de la dentadura completa, maxilares, articulaciones temporomandibulares y estructuras circundantes en una sola toma.

\subsubsection{Principio Físico}
\begin{itemize}
    \item Fuente de rayos X y detector se mueven sincronizadamente
    \item Captura estructuras en un arco curvo
    \item Proyección ortogonal de estructuras tridimensionales en plano 2D
\end{itemize}

\subsection{Aplicaciones Clínicas}

Las ortopantomografías se utilizan para:

\begin{enumerate}
    \item \textbf{Diagnóstico general:}
    \begin{itemize}
        \item Detección de caries
        \item Evaluación de enfermedad periodontal
        \item Identificación de quistes y tumores
    \end{itemize}
    
    \item \textbf{Planificación de tratamientos:}
    \begin{itemize}
        \item Implantes dentales
        \item Cirugía ortognática
        \item Extracción de muelas del juicio
    \end{itemize}
    
    \item \textbf{Evaluación de desarrollo:}
    \begin{itemize}
        \item Erupción dental en niños
        \item Anomalías de desarrollo
    \end{itemize}
\end{enumerate}

\subsection{Limitaciones y Desafíos}

\subsubsection{Limitaciones Técnicas}
\begin{itemize}
    \item Distorsión geométrica inherente
    \item Superposición de estructuras
    \item Variabilidad en exposición
    \item Artefactos de movimiento
\end{itemize}

\subsubsection{Desafíos de Contraste}
\begin{itemize}
    \item Bajo contraste en tejidos blandos
    \item Variabilidad en densidad ósea
    \item Efectos de endurecimiento del haz
    \item Ruido cuántico
\end{itemize}

\section{Procesamiento de Imágenes Médicas}

\subsection{Preprocesamiento}

\subsubsection{Reducción de Ruido}
Técnicas para eliminar ruido mientras se preservan bordes y detalles importantes.

\subsubsection{Normalización}
Ajuste de intensidades para estandarizar imágenes de diferentes fuentes.

\subsection{Mejora de Contraste}

\subsubsection{Ecualización de Histograma Global}
Redistribuye los niveles de intensidad para utilizar todo el rango dinámico.

\[
h(i) = \frac{n_i}{N}
\]

donde $n_i$ es el número de píxeles con intensidad $i$ y $N$ es el número total de píxeles.

\subsubsection{Ecualización de Histograma Adaptativa}
Aplica ecualización en regiones locales de la imagen.

\subsubsection{CLAHE (Contrast Limited Adaptive Histogram Equalization)}
Versión mejorada de AHE que limita la amplificación de contraste para evitar sobre-realce de ruido.

\textbf{Parámetros principales:}
\begin{itemize}
    \item $R_x, R_y$: Tamaño de la región contextual (tile size)
    \item $C$: Límite de contraste (clip limit)
\end{itemize}

\textbf{Algoritmo:}
\begin{algorithm}
\caption{CLAHE}
\begin{algorithmic}
\STATE Dividir imagen en regiones $(R_x \times R_y)$
\FOR{cada región}
    \STATE Calcular histograma local
    \STATE Aplicar límite de contraste $C$
    \STATE Redistribuir píxeles recortados uniformemente
    \STATE Ecualizar histograma
\ENDFOR
\STATE Interpolar entre regiones para eliminar bordes
\end{algorithmic}
\end{algorithm}

\section{Métricas de Evaluación de Calidad}

\subsection{Entropía de Shannon}

Mide la cantidad de información contenida en una imagen:

\[
H = -\sum_{i=0}^{L-1} p_i \log_2(p_i)
\]

donde $p_i$ es la probabilidad del nivel de gris $i$.

\textbf{Interpretación:}
\begin{itemize}
    \item Mayor entropía $\rightarrow$ Mayor información
    \item Rango: $[0, \log_2(L)]$ donde $L$ es el número de niveles de gris
\end{itemize}

\subsection{SSIM (Structural Similarity Index)}

Evalúa la similitud estructural entre dos imágenes:

\[
\text{SSIM}(x,y) = \frac{(2\mu_x\mu_y + C_1)(2\sigma_{xy} + C_2)}{(\mu_x^2 + \mu_y^2 + C_1)(\sigma_x^2 + \sigma_y^2 + C_2)}
\]

donde:
\begin{itemize}
    \item $\mu_x, \mu_y$: medias
    \item $\sigma_x, \sigma_y$: desviaciones estándar
    \item $\sigma_{xy}$: covarianza
    \item $C_1, C_2$: constantes de estabilización
\end{itemize}

\textbf{Componentes:}
\begin{itemize}
    \item Luminancia: $l(x,y)$
    \item Contraste: $c(x,y)$
    \item Estructura: $s(x,y)$
\end{itemize}

\subsection{VQI (Visual Quality Index)}

Métrica perceptual que considera:
\begin{itemize}
    \item Contraste local
    \item Nitidez (sharpness)
    \item Distribución de intensidades
    \item Ausencia de artefactos
\end{itemize}

\section{Estado del Arte}

\subsection{Métodos Tradicionales}

\subsubsection{Mejora Basada en Histograma}
\begin{itemize}
    \item Ecualización global
    \item Especificación de histograma
    \item Ecualización local
\end{itemize}

\subsubsection{Mejora Basada en Dominio de Frecuencia}
\begin{itemize}
    \item Filtrado homómrfico
    \item Filtros de realce de alta frecuencia
\end{itemize}

\subsection{Métodos Avanzados}

\subsubsection{Basados en Optimización}
\begin{itemize}
    \item Algoritmos genéticos para ajuste de parámetros
    \item Optimización por enjambre de partículas
    \item Recocido simulado
\end{itemize}

\subsubsection{Basados en Aprendizaje Profundo}
\begin{itemize}
    \item Redes neuronales convolucionales
    \item Redes adversarias generativas (GANs)
    \item Autoencoders
\end{itemize}

\subsection{Gaps en la Literatura}

A pesar de los avances, existen áreas poco exploradas:
\begin{itemize}
    \item Integración de optimización multiobjetivo con MCDM
    \item Validación sistemática con múltiples métricas
    \item Frameworks flexibles y extensibles
    \item Consideración de preferencias de usuarios expertos
\end{itemize}
