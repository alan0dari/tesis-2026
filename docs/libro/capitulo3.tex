\chapter{Marco Teórico: Optimización Multiobjetivo y MCDM}

\section{Optimización Multiobjetivo}

\subsection{Definición del Problema}

Un problema de optimización multiobjetivo (MOP) se define como:

\begin{align*}
\text{minimizar/maximizar} \quad & \mathbf{F}(\mathbf{x}) = [f_1(\mathbf{x}), f_2(\mathbf{x}), \ldots, f_m(\mathbf{x})]^T \\
\text{sujeto a} \quad & \mathbf{x} \in \Omega
\end{align*}

donde $\mathbf{x} \in \mathbb{R}^n$ es el vector de variables de decisión y $\Omega$ es el espacio de búsqueda factible.

\subsection{Dominancia de Pareto}

Una solución $\mathbf{x}_1$ domina a $\mathbf{x}_2$ ($\mathbf{x}_1 \prec \mathbf{x}_2$) si:
\begin{itemize}
    \item $f_i(\mathbf{x}_1) \leq f_i(\mathbf{x}_2)$ para todo $i \in \{1,\ldots,m\}$
    \item $f_j(\mathbf{x}_1) < f_j(\mathbf{x}_2)$ para al menos un $j \in \{1,\ldots,m\}$
\end{itemize}

\subsection{Frente de Pareto}

El conjunto de soluciones no dominadas forma el \textit{Frente de Pareto}:
\[
\mathcal{P} = \{\mathbf{x} \in \Omega \mid \nexists \mathbf{x}' \in \Omega : \mathbf{x}' \prec \mathbf{x}\}
\]

\section{SMPSO (Speed-constrained Multi-objective PSO)}

\subsection{Particle Swarm Optimization (PSO)}

PSO es un algoritmo de optimización inspirado en el comportamiento social de aves y peces.

\textbf{Ecuaciones de actualización:}
\begin{align}
v_i^{t+1} &= w \cdot v_i^t + c_1 r_1 (p_i - x_i^t) + c_2 r_2 (p_g - x_i^t) \\
x_i^{t+1} &= x_i^t + v_i^{t+1}
\end{align}

\subsection{Extensión Multiobjetivo: SMPSO}

SMPSO introduce:
\begin{enumerate}
    \item \textbf{Restricción de velocidad:} 
    \[
    v_{\max} = \frac{x_{\max} - x_{\min}}{2}
    \]
    
    \item \textbf{Archivo externo:} Mantiene soluciones no dominadas
    
    \item \textbf{Mutación polinomial:} Mantiene diversidad
    
    \item \textbf{Selección de líderes:} Basada en crowding distance
\end{enumerate}

\section{Métodos de Decisión Multicriterio (MCDM)}

\subsection{Problema de Decisión}

Dado un Frente de Pareto con $n$ alternativas y $m$ criterios, seleccionar la mejor alternativa según preferencias del decisor.

\subsection{SMARTER}

\textbf{Idea:} Función de utilidad aditiva con pesos por ranking.

\[
U(A_i) = \sum_{j=1}^m w_j \cdot v_{ij}
\]

Pesos ROC (Rank Order Centroid):
\[
w_j = \frac{1}{m} \sum_{k=j}^m \frac{1}{k}
\]

\subsection{TOPSIS}

\textbf{Idea:} Seleccionar alternativa más cercana a ideal positivo y más lejana de ideal negativo.

\[
C_i = \frac{D_i^-}{D_i^+ + D_i^-}
\]

\subsection{Bellman-Zadeh}

\textbf{Idea:} Decisión difusa como intersección de objetivos.

\[
\mu_D(A_i) = \min_{j} \mu_j(A_i)
\]

\subsection{PROMETHEE II}

\textbf{Idea:} Flujos de preferencia basados en comparaciones pareadas.

\[
\phi(a) = \phi^+(a) - \phi^-(a)
\]

\subsection{GRA (Grey Relational Analysis)}

\textbf{Idea:} Coeficientes de relación gris con respecto a secuencia ideal.

\[
\xi_{ij} = \frac{\Delta_{\min} + \zeta \Delta_{\max}}{\Delta_{ij} + \zeta \Delta_{\max}}
\]

\subsection{VIKOR}

\textbf{Idea:} Índice de compromiso considerando utilidad grupal y arrepentimiento.

\[
Q_i = v \frac{S_i - S^*}{S^- - S^*} + (1-v) \frac{R_i - R^*}{R^- - R^*}
\]

\subsection{CODAS}

\textbf{Idea:} Distancias Euclidiana y Taxicab desde ideal negativo.

\[
H_i = E_i + \sum_k \Psi_{ik} T_i
\]

\subsection{MABAC}

\textbf{Idea:} Distancia al área de aproximación de borde.

\[
S_i = \sum_j (Q_{ij} - G_j)
\]

donde $G_j = (\\prod_i Q_{ij})^{1/n}$ es el BAA.

\section{Métricas de Calidad del Frente de Pareto}

\subsection{Hipervolumen}

Volumen del espacio objetivo dominado por el Frente:
\[
HV = \text{volumen}(\bigcup_{i=1}^{|\mathcal{P}|} \{y \mid y \preceq \mathbf{F}(x_i)\})
\]

\subsection{Spacing}

Uniformidad de distribución:
\[
S = \sqrt{\frac{1}{n}\sum_{i=1}^n (d_i - \bar{d})^2}
\]

\subsection{Spread}

Extensión del Frente:
\[
\Delta = \frac{d_f + d_l + \sum_{i=1}^{n-1}|d_i - \bar{d}|}{d_f + d_l + (n-1)\bar{d}}
\]

\section{Integración de Optimización y MCDM}

\subsection{Flujo General}

\begin{enumerate}
    \item Definir función multiobjetivo
    \item Ejecutar SMPSO para obtener Frente de Pareto
    \item Aplicar múltiples métodos MCDM
    \item Analizar consenso entre métodos
    \item Seleccionar solución final
\end{enumerate}

\subsection{Ventajas de la Integración}

\begin{itemize}
    \item Exploración completa del espacio de soluciones
    \item Múltiples perspectivas de selección
    \item Robustez mediante consenso
    \item Flexibilidad en preferencias
\end{itemize}
