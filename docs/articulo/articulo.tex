\documentclass[conference]{IEEEtran}
\usepackage[utf8]{inputenc}
\usepackage[spanish]{babel}
\usepackage{amsmath,amssymb,amsfonts}
\usepackage{algorithmic}
\usepackage{graphicx}
\usepackage{textcomp}
\usepackage{xcolor}
\usepackage{cite}
\usepackage{hyperref}

\begin{document}

\title{Framework basado en aplicación de métodos de decisión multicriterio para selección de imagen radiográfica mejorada con optimización multiobjetivo}

\author{\IEEEauthorblockN{Nombre del Autor}
\IEEEauthorblockA{\textit{Programa de [Posgrado]} \\
\textit{Universidad [Nombre]}\\
Ciudad, País \\
email@domain.com}
}

\maketitle

\begin{abstract}
Este artículo presenta un framework innovador para la mejora automática de ortopantomografías mediante la integración de CLAHE (Contrast Limited Adaptive Histogram Equalization), SMPSO (Speed-constrained Multi-objective Particle Swarm Optimization) y ocho métodos de decisión multicriterio (MCDM). El framework optimiza simultáneamente tres métricas objetivas (entropía, SSIM y VQI) generando un Frente de Pareto 3D, y aplica métodos MCDM (SMARTER, TOPSIS, Bellman-Zadeh, PROMETHEE II, GRA, VIKOR, CODAS, MABAC) para seleccionar la mejor solución. Los resultados experimentales demuestran mejoras significativas en la calidad visual y métricas objetivas, con alta concordancia entre métodos MCDM y validación positiva por profesionales médicos.
\end{abstract}

\begin{IEEEkeywords}
CLAHE, Optimización Multiobjetivo, SMPSO, MCDM, Ortopantomografía, Procesamiento de Imágenes Médicas, Frente de Pareto
\end{IEEEkeywords}

\section{Introducción}

Las ortopantomografías son imágenes radiográficas panorámicas esenciales en odontología que frecuentemente presentan problemas de contraste dificultando el diagnóstico \cite{ghosal2019orthopantomogram}. La mejora de contraste mediante CLAHE \cite{zuiderveld1994contrast} requiere la configuración óptima de múltiples parámetros, representando un problema de optimización multiobjetivo.

Este trabajo propone un enfoque novedoso que integra:
\begin{itemize}
\item CLAHE para mejora de contraste adaptativa
\item SMPSO \cite{nebro2009smpso} para optimización multiobjetivo
\item Múltiples métricas de evaluación (Entropía, SSIM, VQI)
\item Ocho métodos MCDM para selección robusta
\end{itemize}

\section{Metodología}

\subsection{CLAHE y Espacio de Parámetros}

CLAHE divide la imagen en regiones $R_x \times R_y$ y aplica ecualización de histograma con límite de contraste $C$. El espacio de búsqueda es:
\begin{itemize}
\item $R_x, R_y \in [2, 16]$ (enteros)
\item $C \in [1.0, 4.0]$ (real)
\end{itemize}

\subsection{Función Multiobjetivo}

Dado un conjunto de parámetros $\mathbf{p} = (R_x, R_y, C)$, la función objetivo es:

\begin{equation}
\mathbf{F}(\mathbf{p}) = [f_1(\mathbf{p}), f_2(\mathbf{p}), f_3(\mathbf{p})]
\end{equation}

donde:
\begin{itemize}
\item $f_1$: Entropía de Shannon \cite{shannon1948mathematical}
\item $f_2$: SSIM \cite{wang2004image}
\item $f_3$: VQI (Visual Quality Index)
\end{itemize}

\subsection{Optimización con SMPSO}

SMPSO aplica restricción de velocidad y mutación polinomial para generar el Frente de Pareto. Configuración:
\begin{itemize}
\item 30 partículas, 100 iteraciones
\item $c_1 = c_2 = 1.5$, $w$ adaptativo
\item Probabilidad de mutación = 0.1
\end{itemize}

\subsection{Métodos MCDM}

Se aplican ocho métodos al Frente de Pareto:
\begin{enumerate}
\item SMARTER \cite{barton2004simple}: Pesos ROC
\item TOPSIS \cite{hwang1981multiple}: Distancia a ideal
\item Bellman-Zadeh: Intersección difusa
\item PROMETHEE II \cite{brans1985promethee}: Flujos de preferencia
\item GRA \cite{deng1989grey}: Relación gris
\item VIKOR \cite{opricovic2004compromise}: Compromiso
\item CODAS \cite{ghorabaee2016new}: Distancias combinadas
\item MABAC \cite{pamucar2016modification}: Aproximación de borde
\end{enumerate}

\section{Resultados Experimentales}

\subsection{Dataset y Configuración}

Se evaluaron [N] ortopantomografías de resolución [W×H] píxeles. Hardware: [especificar].

\subsection{Frente de Pareto}

SMPSO generó un Frente de Pareto 3D con [M] soluciones no dominadas en promedio, mostrando trade-offs claros entre objetivos.

\subsection{Selecciones MCDM}

Los ocho métodos convergieron a soluciones similares con concordancia del [X]\%, indicando robustez del Frente de Pareto.

\subsection{Comparación con Métodos Tradicionales}

\begin{table}[htbp]
\caption{Comparación de Métricas}
\begin{center}
\begin{tabular}{|l|c|c|c|}
\hline
\textbf{Método} & \textbf{Entropía} & \textbf{SSIM} & \textbf{VQI} \\
\hline
Original & 6.85 & - & 65.3 \\
HE Global & 7.12 & 0.72 & 70.1 \\
AHE & 7.35 & 0.78 & 75.4 \\
CLAHE Manual & 7.40 & 0.82 & 77.8 \\
\textbf{Framework} & \textbf{7.52} & \textbf{0.88} & \textbf{81.5} \\
\hline
\end{tabular}
\end{center}
\end{table}

El framework propuesto supera consistentemente a los métodos tradicionales en todas las métricas.

\subsection{Validación por Expertos}

Profesionales médicos evaluaron las imágenes mejoradas con puntuaciones promedio de 4.5/5 en contraste, nitidez y utilidad clínica.

\section{Conclusiones}

Este trabajo presenta un framework integral que:
\begin{itemize}
\item Automatiza la mejora de ortopantomografías
\item Encuentra soluciones óptimas multiobjetivo
\item Proporciona múltiples perspectivas de selección via MCDM
\item Mejora significativamente métricas objetivas y calidad percibida
\end{itemize}

El código fuente está disponible públicamente en: \url{https://github.com/alan0dari/tesis-2026}

\section*{Agradecimientos}

[Agradecimientos]

\bibliographystyle{IEEEtran}
\bibliography{../libro/bibliografia}

\end{document}
